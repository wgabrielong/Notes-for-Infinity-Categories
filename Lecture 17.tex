\section{Lecture 17 -- 13th November 2023}
In this lecture and in the remainder of this course we will consider examples of $\infty$-categories, developing whatever further formalisms of $\infty$-categories we may need on the fly. We will try to embody a philosophy apocryphally attributed to Grothendieck, allowing the mathematical structure to tell us what it wants to be. 
\\\\
We begin by exploring simplicially enriched categories, which were first studied by Dan Kan, and are related to simplicial categories. Recall for $X,Y$ simplicial sets, we had $\Mor_{\SSets}(X,Y)=Y^{X}$ a simplicial set of maps from $X$ to $Y$. Naturally, there is an identity map on $X$, $\id_{X}$ which we can view as the map of a 0-simplex to a mapping space $\id_{X}:\Delta^{0}\to X^{X}$. For simplicial sets $X,Y,Z$, there is naturally a composition law $Z^{Y}\times Y^{X}\to Z^{X}$ that is unital and associative. This leads to the following definition of a simplicially enriched category. 
\begin{definition}[Simplicially Enriched Category]\label{def: simplicially enriched category}
    A simplicially enriched category $\Csf$ consists of a class $\Obj(\Csf)$ and for $A,B\in\Obj(\Csf)$ a simplical set $\Mor_{\Csf}(A,B)$ such that 
    \begin{enumerate}[label=(\alph*)]
        \item there is a map $\id_{A}:\Delta^{0}\to\Mor_{\Csf}(A,A)$, 
        \item and a composition law $\Mor_{\Csf}(B,C)\times\Mor_{\Csf}(A,B)\to\Mor_{\Csf}(A,C)$ that is unital and associative. 
    \end{enumerate}
\end{definition}
Note that in a simplicially enriched category, morphisms have a higher structure. For a commuting diagram 
$$% https://q.uiver.app/#q=WzAsMyxbMCwwLCJBIl0sWzIsMCwiQiJdLFsxLDEsIkMiXSxbMCwyLCJoIiwyXSxbMCwxLCJmIl0sWzEsMiwiZyJdXQ==
\begin{tikzcd}
	A && B \\
	& C
	\arrow["h"', from=1-1, to=2-2]
	\arrow["f", from=1-1, to=1-3]
	\arrow["g", from=1-3, to=2-2]
\end{tikzcd}$$
we have maps $f:\Delta^{0}\to\Mor_{\Csf}(A,B), g:\Delta^{0}\to\Mor_{\Csf}(B,C)$ and $\tau:\Delta^{1}\to\Mor_{\Csf}(A,C)$ such that $\tau\langle0\rangle = h$ and $\tau\langle 1\rangle = g\circ f$. One naturally defines a functor of simplicially enriched categories as a map on objects and a simplicial set map on morphisms preserving units and composition. 
\\\\
Denote $\Cat_{\DDelta}$ the 2-category of simplicially enriched categories. We want to construct a functor $\Cat_{\DDelta}\to\SSets$. We will do this via the homotopy coherent nerve. We first consider the following construction of a path category. 
\begin{definition}[Path Category]\label{def: path category}
    Let $(S,\leq)$ be a poset. The path category on $S$ $\Path(S)$ is a category with objects those of $S$ and for $s,t\in S$, $\Mor_{\Path(S)}(s,t)$ is the nerve of the finite linearly ordered set between $s$ and $t$. 
\end{definition}
\begin{example}
    Consider $\Path([1])$ where we have $\Mor_{\Path(S)}(0,1)=\{*\}$ so the simplicial set has two 0-simplices with a 1-simplex connecting them. 
\end{example}
\begin{example}
    For $\Path([2])$ we have the following. 
    $$% https://q.uiver.app/#q=WzAsMyxbMCwwLCIwIl0sWzIsMCwiMiJdLFsxLDEsIjEiXSxbMCwyXSxbMiwxXSxbMCwxLCJcXHsxXFx9IiwxLHsiY3VydmUiOi0yfV0sWzAsMSwiXFxlbXB0eXNldCIsMSx7ImN1cnZlIjoyfV0sWzUsNiwiIiwyLHsic2hvcnRlbiI6eyJzb3VyY2UiOjIwLCJ0YXJnZXQiOjIwfX1dXQ==
    \begin{tikzcd}
        0 && 2 \\
        & 1
        \arrow[from=1-1, to=2-2]
        \arrow[from=2-2, to=1-3]
        \arrow[""{name=0, anchor=center, inner sep=0}, "{\{1\}}"{description}, curve={height=-12pt}, from=1-1, to=1-3]
        \arrow[""{name=1, anchor=center, inner sep=0}, "\emptyset"{description}, curve={height=12pt}, from=1-1, to=1-3]
        \arrow[shorten <=3pt, shorten >=3pt, Rightarrow, from=0, to=1]
    \end{tikzcd}$$
\end{example}
This allows us to define the homotopy coherent nerve of a simplicially enriched category. 
\begin{definition}[Homotopy Coherent Nerve]
    Let $\Csf$ be a simplicially enriched category. The homotopy coherent nerve $N^{\mathrm{hc}}\Csf$ is such that $(N^{\mathrm{hc}}\Csf)_{n}=\Mor_{\Cat_{\DDelta}}(\Path([n]),\Csf)$ such that the simplicial operators act by precomposition. 
\end{definition}
One naturally wonders if $N^{\mathrm{hc}}\Csf$ is a quaiscategory. It turns out this is not always the case. In fact, one can show the following. 
\begin{theorem}[Cordier and Porter]\label{thm: Cordier and Porter}
    Let $\Csf$ be a simplicially enriched category. If $\Mor_{\Csf}(A,B)$ is a Kan complex for all $A,B\in\Obj(\Csf)$, then $N^{\mathrm{hc}}\Csf$ is a quasicategory. 
\end{theorem}
Let $\Kan$ be the full subcategory of simplicial sets $\SSets$ containing all Kan complexes. But wa also defined $\Kan_{\infty}$ in \Cref{def: kan-infty} which was the base of the universal left fibration. These two objects are related by the following theorem. 
\begin{theorem}[Lurie]\label{thm: kan-infty and hc nerve of kan are equivalent}
    $\Kan_{\infty}$ and $N^{\mathrm{hc}}\Kan$ are equivalent. 
\end{theorem}
One proves this by showing that $N^{\mathrm{hc}}(\Kan)$ is the base of the universal left fibration. This is known as straightening and unstraightening, more information on which can be found in \cite[Ch. 4]{Land} and \cite[Ch. 3.2]{LurieHTT}. One can also make a similar construction to show that for $\mathsf{QCat}$ the quasicategory of quaiscategories, $N^{\mathrm{hc}}\mathsf{QCat}$ and $\Cat_{\infty}$ are equivalent. 
\\\\
Let $I$ be a category and consider $\SPSh(I)$, functors $I^{\Opp}\to\SSets$. Otherwise put, $\SPSh(I)$ is the category of presheaves of simplicial sets on $I$. Restricting our attention to presheaves on $I$ that take values in Kan complexes instead of arbitrary simplicial sets, this category is $N^{\mathrm{hc}}\SPSh(I)$. \\\\
Maps $N^{\mathrm{hc}}\SPSh(I)\to X$ for $X$ a cocomplete quasicategory correspond to $I\to X$ in $\Cat_{\infty}$, that is, a coCartesian fibration over $\Delta^{1}$ restricting to $I$ on one end and $X$ on the other. \\\\
More generally, in the topological category, we define a presheaf of sets on a topoological space $X$ as follows. 
\begin{definition}[Presheaf on a Topological Space]\label{def: presheaf on topological space}
    Let $X$ be a topological space and $X^{\Opens}$ the category whose objects are open sets of $X$ and whose morphisms are inclusions. A presheaf of sets on $X$ is a functor $\Fcal:(X^{\Opens})^{\Opp}\to \Sets$. 
\end{definition}
The prototypical example here is of continuous functions on a manifold $X$. In such a situation, continuous functions glue: for $f_{1}$ on $U_{1}$ and $f_{2}$ on $U_{2}$ continuous such that $f_{1}|_{U_{1}\cap U_{2}}=f_{2}|_{U_{1}\cap U_{2}}$ then there is a continuous function on $U_{1}\cup U_{2}$ restricting to $f_{1}$ and $f_{2}$ on $U_{1}$ and $U_{2}$, respectively. This leads to the definition of a sheaf. 
\begin{definition}[Sheaf]\label{def: sheaf on topological space}
    Let $X$ be a topological space and $X^{\Opens}$ the category whose objects are open sets of $X$ and whose morphisms are inclusions. A presheaf of sets $\Fcal:(X^{\Opens})^{\Opp}\to\Sets$ is a sheaf if the sequence 
    $$% https://q.uiver.app/#q=WzAsMyxbMCwwLCJcXEZjYWwoWCkiXSxbMiwwLCJcXHByb2Rfe2l9XFxGY2FsKFhfe2l9KSJdLFs0LDAsIlxccHJvZF97aSxqfVxcRmNhbChYX3tpfVxcY2FwIFhfe2p9KSJdLFswLDFdLFsxLDIsIiIsMCx7Im9mZnNldCI6LTF9XSxbMSwyLCIiLDAseyJvZmZzZXQiOjF9XV0=
    \begin{tikzcd}
        {\Fcal(X)} && {\prod_{i}\Fcal(X_{i})} && {\prod_{i,j}\Fcal(X_{i}\cap X_{j})}
        \arrow[from=1-1, to=1-3]
        \arrow[shift left, from=1-3, to=1-5]
        \arrow[shift right, from=1-3, to=1-5]
    \end{tikzcd}$$
    is an equalizer for $\{X_{i}\}$ an open cover of $X$. 
\end{definition}
In this sense, we say that a sheaf is a presheaf satisfying descent. 
\\\\
This leads to the definition of localization, where we force certain maps to be isomorphisms. We consider the definition of localization in quasicategories.\newpage 
\begin{definition}[Localization in Quasicategories]\label{def: localization of quasicategories}
    Let $X$ be a quaiscategory and $S\subseteq X$ a simplicial subset. The localization of $X$ at $S$ denoted $X[S^{-1}]$ is such that for every quasicategory $Y$ the map $Y^{X[S^{-1}]}\to (Y^{\sim})^{S}\times_{Y}Y^{X}$ in the diagram 
    $$% https://q.uiver.app/#q=WzAsNSxbMCwwLCJZXntYW1Neey0xfV19Il0sWzEsMSwiKFlee1xcc2ltfSlee1N9XFx0aW1lc197WX1ZXntYfSJdLFsxLDIsIihZXntcXHNpbX0pXntTfSJdLFszLDEsIllee1h9Il0sWzMsMiwiWV57U30iXSxbMCwxLCIiLDAseyJzdHlsZSI6eyJib2R5Ijp7Im5hbWUiOiJkYXNoZWQifX19XSxbMCwzLCIiLDIseyJjdXJ2ZSI6LTJ9XSxbMCwyLCIiLDIseyJjdXJ2ZSI6Mn1dLFsyLDRdLFszLDRdLFsxLDJdLFsxLDNdXQ==
    \begin{tikzcd}
        {Y^{X[S^{-1}]}} \\
        & {(Y^{\sim})^{S}\times_{Y}Y^{X}} && {Y^{X}} \\
        & {(Y^{\sim})^{S}} && {Y^{S}}
        \arrow[dashed, from=1-1, to=2-2]
        \arrow[curve={height=-12pt}, from=1-1, to=2-4]
        \arrow[curve={height=12pt}, from=1-1, to=3-2]
        \arrow[from=3-2, to=3-4]
        \arrow[from=2-4, to=3-4]
        \arrow[from=2-2, to=3-2]
        \arrow[from=2-2, to=2-4]
    \end{tikzcd}$$
    is an acyclic Kan fibration. 
\end{definition}
We can consider the following examples. 
\begin{example}
    Let $\Csf$ be a category with a Grothendieck topology and $S$ the collection of nerves of covers of $\Csf$. We can consider the localization 
    $$N^{\mathrm{hc}}(\SPSh(\Csf))[S^{-1}]$$
    the homotopy coherent sheaf theory on $\Csf$. 
\end{example}
\begin{example}
    Let $k$ be a field and $\mathsf{SmSch}_{k}$ the category of smooth quasiprojective schemes over $k$. We consider the localization over Nisnevich covers (arithmetically trivial maps) and that arise as fibers of a higher-dimensional schemes over $\mathbb{A}^{1}$
    $$\SPSh(\mathsf{SmSch}_{k})[\mathsf{Nis}^{-1}, \{M\times\{0\}\to M\times\mathbb{A}^{1}_{k}\}^{-1}]$$
    the $\mathbb{A}^{1}$-homotopy category of schemes. This forces a notion of $\mathbb{A}^{1}$-equivalence similar to the notion of rational equivalence on schemes. 
\end{example}