\section{Lecture 11 -- 23rd October 2023}
We are trying to think about mathematics inside a quaicategory, and define constructions such as initial and final objects, limits and colimits, and more. Let us consider something that we have yet to develop the language to prove. 
\\\\
Suppose $F:I\to\Csf$ be a functor from an indexing category $I$ to a category $\Csf$. Let $G:\Csf\to\Dsf$ be a functor. When is $G\left(\colim_{I}F(i)\right)=\colim_{I}(G\circ F)(i)$? Evidently, this holds if $G$ is an equivalence of categories, but in what other settings does it hold? What is the analogue in the setting of quasicategories?
\\\\
We will consider isomorphisms in quasicategories by proving the following theorem of Joyal. 
\begin{theorem}[Joyal]\label{thm: Joyal on iso iff iso in fundamental category}
    Let $X$ be a quasicategory and $f\in X_{1}$. If the image of $f$ is an isomorphism in the fundamental category of $X$, then $f$ is an isomorphism in $X$. 
\end{theorem}
Recall that to define the homotopy category we had to consider left and right inverses that were witnessed by 2-simplices. For $f\in X_{1}$ with $f:A\to B$ and $A,B\in X_{0}$, we say that $f$ admits a left inverse $g$ if the diagram 
$$% https://q.uiver.app/#q=WzAsMyxbMCwwLCJBIl0sWzAsMSwiQSJdLFsyLDEsIkIiXSxbMCwxLCJcXHdyIiwyXSxbMSwyLCJmIiwyXSxbMiwwLCJnIiwyXV0=
\begin{tikzcd}
	A \\
	A && B
	\arrow["\wr"', from=1-1, to=2-1]
	\arrow["f"', from=2-1, to=2-3]
	\arrow["g"', from=2-3, to=1-1]
\end{tikzcd}$$
commutes where $g\circ f\simeq\id_{A}$. Similarly, we say $f$ admits a right inverse $g'$ if the diagram 
$$% https://q.uiver.app/#q=WzAsMyxbMCwxLCJBIl0sWzIsMSwiQiJdLFsyLDAsIkIiXSxbMSwyLCJcXHdyIiwyXSxbMiwwLCJnJyIsMl0sWzAsMSwiZiIsMl1d
\begin{tikzcd}
	&& B \\
	A && B
	\arrow["\wr"', from=2-3, to=1-3]
	\arrow["{g'}"', from=1-3, to=2-1]
	\arrow["f"', from=2-1, to=2-3]
\end{tikzcd}$$
commutes where $f\circ g'=\id_{B}$. 
\begin{remark}
    In the setting of an ordinary category, we can verify 
    $$g=g\circ\id_{B}=g\circ f\circ g'=\id_{A}\circ g'=g'$$
    showing that $g=g'$. 
\end{remark}
We now state another theorem of Joyal, known as the Joyal Extension theorem, which will imply the statement in \Cref{thm: Joyal on iso iff iso in fundamental category}. 
\begin{theorem}[Joyal -- Extension]\label{thm: Joyal extension theorem}
    Let $X$ be a quasicategory and $f\in X_{1}$. The following are equivalent. 
    \begin{enumerate}[label=(\alph*)]
        \item $f$ is an isomorphism. 
        \item The horn inclusion of $f$ into the leading edge of a horn can be lifted. 
        $$% https://q.uiver.app/#q=WzAsNCxbMCwwLCJcXERlbHRhXnsxfSJdLFs0LDAsIlgiXSxbMiwwLCJcXExhbWJkYV97MH1ee259Il0sWzIsMSwiXFxEZWx0YV57bn0iXSxbMywxLCJcXGV4aXN0cyIsMSx7InN0eWxlIjp7ImJvZHkiOnsibmFtZSI6ImRhc2hlZCJ9fX1dLFsyLDFdLFsyLDNdLFswLDIsIlxcbGFuZ2xlMCwxXFxyYW5nbGUiLDJdLFswLDEsImYiLDEseyJjdXJ2ZSI6LTN9XV0=
        \begin{tikzcd}
            {\Delta^{1}} && {\Lambda_{0}^{n}} && X \\
            && {\Delta^{n}}
            \arrow["\exists"{description}, dashed, from=2-3, to=1-5]
            \arrow[from=1-3, to=1-5]
            \arrow[from=1-3, to=2-3]
            \arrow["{\langle0,1\rangle}"', from=1-1, to=1-3]
            \arrow["f"{description}, curve={height=-18pt}, from=1-1, to=1-5]
        \end{tikzcd}$$
        \item The horn inclusion of $f$ into the tailing edge of a horn can be lifted. 
        $$% https://q.uiver.app/#q=WzAsNCxbMCwwLCJcXERlbHRhXnsxfSJdLFs0LDAsIlgiXSxbMiwwLCJcXExhbWJkYV97bn1ee259Il0sWzIsMSwiXFxEZWx0YV57bn0iXSxbMywxLCJcXGV4aXN0cyIsMSx7InN0eWxlIjp7ImJvZHkiOnsibmFtZSI6ImRhc2hlZCJ9fX1dLFsyLDFdLFsyLDNdLFswLDIsIlxcbGFuZ2xlIG4tMSxuXFxyYW5nbGUiLDJdLFswLDEsImYiLDEseyJjdXJ2ZSI6LTN9XV0=
        \begin{tikzcd}
            {\Delta^{1}} && {\Lambda_{n}^{n}} && X \\
            && {\Delta^{n}}
            \arrow["\exists"{description}, dashed, from=2-3, to=1-5]
            \arrow[from=1-3, to=1-5]
            \arrow[from=1-3, to=2-3]
            \arrow["{\langle n-1,n\rangle}"', from=1-1, to=1-3]
            \arrow["f"{description}, curve={height=-18pt}, from=1-1, to=1-5]
        \end{tikzcd}$$
    \end{enumerate}
\end{theorem}
The proof of this theorem will take some additional tools that we have yet to discuss. Following the correspondence between isomorphisms in quasicategories and isomorphisms in categories, we define the following. 
\begin{definition}[Conservative Functor]\label{def: conservative functor}
    Let $F:X\to Y$ be a functor between quasicategories. The functor $F$ is conservative if for all $f\in X_{1}$ and $F(f)\in Y_{1}$ is an isomorphism then $f$ is an isomorphism in $X$. 
\end{definition}
\begin{remark}
    This is a very important concept. In fact, some conjectures in algebraic geometry reduce to showing a functor is conservative. 
\end{remark}
\begin{proposition}
    If $F$ is a left (resp. right) fibration then $F$ is conservative. 
\end{proposition}
\begin{proof}
    We prove the case of left fibrations. 
    \\\\
    Suppose that $F:X\to Y$ is a left fibration, $f\in X_{1}, F(f)\in Y_{1}$ is an isomorphism. Choose $\tau\in Y_{2}$ witnessing the inverse $g$ of $F(f)$. 
    $$% https://q.uiver.app/#q=WzAsMyxbMCwxLCIwIl0sWzIsMSwiMSJdLFsxLDAsIjIiXSxbMCwxLCJGKGYpIiwxXSxbMSwyLCJnIiwxXSxbMCwyLCJcXHNpbSJdXQ==
    \begin{tikzcd}
        & 2 \\
        0 && 1
        \arrow["{F(f)}"{description}, from=2-1, to=2-3]
        \arrow["g"{description}, from=2-3, to=1-2]
        \arrow["\sim", from=2-1, to=1-2]
    \end{tikzcd}$$
    We now consider the lifting problem 
    $$% https://q.uiver.app/#q=WzAsNCxbMCwwLCJcXExhbWJkYV57Mn1fezB9Il0sWzAsMSwiXFxEZWx0YV57Mn0iXSxbMiwxLCJZIl0sWzIsMCwiWCJdLFswLDFdLFszLDIsIkYiXSxbMCwzLCJcXHNpZ21hIl0sWzEsMiwiXFx0YXUiLDJdLFsxLDMsIlxcd2lkZXRpbGRle1xcdGF1fSIsMSx7InN0eWxlIjp7ImJvZHkiOnsibmFtZSI6ImRhc2hlZCJ9fX1dXQ==
    \begin{tikzcd}
        {\Lambda^{2}_{0}} && X \\
        {\Delta^{2}} && Y
        \arrow[from=1-1, to=2-1]
        \arrow["F", from=1-3, to=2-3]
        \arrow["\sigma", from=1-1, to=1-3]
        \arrow["\tau"', from=2-1, to=2-3]
        \arrow["{\widetilde{\tau}}"{description}, dashed, from=2-1, to=1-3]
    \end{tikzcd}$$
    where $\sigma\langle0,2\rangle=\id, \sigma\langle0,1\rangle=f$ and the lift $\widetilde{\tau}$ exists by $F$ being a left fibration where we take $\widetilde{\tau}\langle1,2\rangle$ to be the left inverse of $f$ showing that $f$ is an isomorphism. 
\end{proof}
We can now define isofibrations. 
\begin{definition}[Isofibration; Riehl-Verity, Lurie]\label{def: isofibration}
    A functor $F:X\to Y$ between quasicategories is an isofibration if it is an inner fibration and for every diagram of the form
    $$% https://q.uiver.app/#q=WzAsNCxbMCwwLCJcXERlbHRhXnswfSJdLFswLDEsIlxcRGVsdGFeezF9Il0sWzIsMCwiWCJdLFsyLDEsIlkiXSxbMiwzLCJGIl0sWzAsMl0sWzEsMywiZiIsMl0sWzAsMSwiXFxsYW5nbGUwXFxyYW5nbGUiLDJdLFsxLDIsImciLDEseyJzdHlsZSI6eyJib2R5Ijp7Im5hbWUiOiJkYXNoZWQifX19XV0=
    \begin{tikzcd}
        {\Delta^{0}} && X \\
        {\Delta^{1}} && Y
        \arrow["F", from=1-3, to=2-3]
        \arrow[from=1-1, to=1-3]
        \arrow["f"', from=2-1, to=2-3]
        \arrow["\langle0\rangle"', from=1-1, to=2-1]
        \arrow["g"{description}, dashed, from=2-1, to=1-3]
    \end{tikzcd}$$
    $f$ being an isomorphism implies $g$ being an isomorphism. 
\end{definition}
In fact, we can show that the lifting the tail in the definition above can be rephrased as lifting a tip. 
\begin{lemma}
    Let $F:X\to Y$ be an inner fibration between quasicategories. The diagram 
    $$% https://q.uiver.app/#q=WzAsNCxbMCwwLCJcXERlbHRhXnswfSJdLFswLDEsIlxcRGVsdGFeezF9Il0sWzIsMCwiWCJdLFsyLDEsIlkiXSxbMiwzLCJGIl0sWzAsMl0sWzEsMywiZiIsMl0sWzAsMSwiXFxsYW5nbGUwXFxyYW5nbGUiLDJdLFsxLDIsImciLDEseyJzdHlsZSI6eyJib2R5Ijp7Im5hbWUiOiJkYXNoZWQifX19XV0=
    \begin{tikzcd}
        {\Delta^{0}} && X \\
        {\Delta^{1}} && Y
        \arrow["F", from=1-3, to=2-3]
        \arrow[from=1-1, to=1-3]
        \arrow["f"', from=2-1, to=2-3]
        \arrow["\langle0\rangle"', from=1-1, to=2-1]
        \arrow["g"{description}, dashed, from=2-1, to=1-3]
    \end{tikzcd}$$
    admits a lift if and only if the diagram 
    $$% https://q.uiver.app/#q=WzAsNCxbMCwwLCJcXERlbHRhXnswfSJdLFswLDEsIlxcRGVsdGFeezF9Il0sWzIsMCwiWCJdLFsyLDEsIlkiXSxbMiwzLCJGIl0sWzAsMl0sWzEsMywiZiIsMl0sWzAsMSwiXFxsYW5nbGUxXFxyYW5nbGUiLDJdLFsxLDIsImciLDEseyJzdHlsZSI6eyJib2R5Ijp7Im5hbWUiOiJkYXNoZWQifX19XV0=
    \begin{tikzcd}
        {\Delta^{0}} && X \\
        {\Delta^{1}} && Y
        \arrow["F", from=1-3, to=2-3]
        \arrow[from=1-1, to=1-3]
        \arrow["f"', from=2-1, to=2-3]
        \arrow["\langle1\rangle"', from=1-1, to=2-1]
        \arrow["g"{description}, dashed, from=2-1, to=1-3]
    \end{tikzcd}$$
    does. 
\end{lemma}
We can then show the following. 
\begin{proposition}
    Let $F:X\to Y$ be a functor between quasicategories. If $F$ is a left or right fibration then $F$ is an isofibration. 
\end{proposition}