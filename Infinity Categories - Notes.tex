\documentclass[]{scrarticle}
\usepackage[left=2cm,right=2cm,top=2.5cm,bottom=2.5cm]{geometry}
\usepackage{arsclassica}
\usepackage{classicthesis}
\usepackage{amsmath, amssymb, tikz, amsthm, csquotes, multicol, footnote, tablefootnote, biblatex, wrapfig, float, epigraph, hyperref}
\newtheorem{theorem}{Theorem}[section]
\newtheorem{corollary}{Corollary}[theorem]
\newtheorem{lemma}[theorem]{Lemma}
\newtheorem{proposition}[theorem]{Proposition}
\theoremstyle{definition}
\newtheorem{definition}{Definition}[section]
\theoremstyle{remark}
\newtheorem*{remark}{Remark}
\theoremstyle{example}
\newtheorem{example}{Example}

\title{MATH 292Z: First Steps in Infinity Categories}
\author{\spacedlowsmallcaps{Wern Juin Gabriel Ong}}
\date{}
\begin{document}
\maketitle
\begin{abstract}
    These notes roughly correspond to the course \texttt{MATH 292Z: First Steps in Infinity Categories} taught by Prof. Michael Hopkins at Harvard University in the Fall 2023 semester. These notes are \LaTeX-ed after the fact with significant alteration and are subject to misinterpretation and mistranscription. Use with caution. Any errors are undoubtedly my own and any virtues that could be ascribed to these notes ought be attributed to the instructor and not the typist. 
\end{abstract}
\tableofcontents
\section{Lecture 1}
\end{document}