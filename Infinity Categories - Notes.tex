\documentclass{amsart}
\usepackage[margin=1.1in]{geometry} 
\usepackage{amsmath}
\usepackage{tcolorbox}
\usepackage{amssymb}
\usepackage{amsthm}
\usepackage{lastpage}
\usepackage{fancyhdr}
\usepackage{accents}
\usepackage{hyperref}
\usepackage{xcolor}
\usepackage{color}
% Fields
\newcommand{\CC}{\mathbb{C}}
\newcommand{\RR}{\mathbb{R}}
\newcommand{\QQ}{\mathbb{Q}}
\newcommand{\ZZ}{\mathbb{Z}}
\newcommand{\NN}{\mathbb{N}}
\newcommand{\FF}{\mathbb{F}}
\newcommand{\PP}{\mathbb{P}}

% mathcal letters
\newcommand{\Acal}{\mathcal{A}}
\newcommand{\Bcal}{\mathcal{B}}
\newcommand{\Ccal}{\mathcal{C}}
\newcommand{\Dcal}{\mathcal{D}}
\newcommand{\Ecal}{\mathcal{E}}
\newcommand{\Fcal}{\mathcal{F}}
\newcommand{\Gcal}{\mathcal{G}}
\newcommand{\Hcal}{\mathcal{H}}
\newcommand{\Ical}{\mathcal{I}}
\newcommand{\Jcal}{\mathcal{J}}
\newcommand{\Kcal}{\mathcal{K}}
\newcommand{\Lcal}{\mathcal{L}}
\newcommand{\Mcal}{\mathcal{M}}
\newcommand{\Ncal}{\mathcal{N}}
\newcommand{\Ocal}{\mathcal{O}}
\newcommand{\Pcal}{\mathcal{P}}
\newcommand{\Qcal}{\mathcal{Q}}
\newcommand{\Rcal}{\mathcal{R}}
\newcommand{\Scal}{\mathcal{S}}
\newcommand{\Tcal}{\mathcal{T}}
\newcommand{\Ucal}{\mathcal{U}}
\newcommand{\Vcal}{\mathcal{V}}
\newcommand{\Wcal}{\mathcal{W}}
\newcommand{\Xcal}{\mathcal{X}}
\newcommand{\Ycal}{\mathcal{Y}}
\newcommand{\Zcal}{\mathcal{Z}}

% abstract categories
\newcommand{\Asf}{\mathsf{A}}
\newcommand{\Bsf}{\mathsf{B}}
\newcommand{\Csf}{\mathsf{C}}
\newcommand{\Dsf}{\mathsf{D}}
\newcommand{\Esf}{\mathsf{E}}
\newcommand{\Ssf}{\mathsf{S}}

% algebraic geometry
\newcommand{\spec}{\operatorname{Spec}}
\newcommand{\proj}{\operatorname{Proj}}

% categories 
\newcommand{\id}{\mathrm{id}}
\newcommand{\Obj}{\mathrm{Obj}}
\newcommand{\Mor}{\mathrm{Mor}}
\newcommand{\Hom}{\mathrm{Hom}}
\newcommand{\Aut}{\mathrm{Aut}}
\newcommand{\Sets}{\mathsf{Sets}}
\newcommand{\SSets}{\mathsf{SSets}}
\newcommand{\kVec}{\mathsf{Vec}_{k}}
\newcommand{\Alg}{\mathsf{Alg}}
\newcommand{\Ring}{\mathsf{Ring}}
\newcommand{\Mod}{\mathsf{Mod}}
\newcommand{\Grp}{\mathsf{Grp}}
\newcommand{\AbGrp}{\mathsf{AbGrp}}
\newcommand{\PSh}{\mathsf{PSh}}
\newcommand{\Sh}{\mathsf{Sh}}
\newcommand{\PSch}{\mathsf{PSch}}
\newcommand{\Sch}{\mathsf{Sch}}
\newcommand{\Top}{\mathsf{Top}}
\newcommand{\Com}{\mathsf{Com}}
\newcommand{\Coh}{\mathsf{Coh}}
\newcommand{\QCoh}{\mathsf{QCoh}}
\newcommand{\Opens}{\mathsf{Opens}}
\newcommand{\Opp}{\mathsf{Opp}}
\newcommand{\Cat}{\mathsf{Cat}}
\newcommand{\NatTrans}{\mathrm{NatTrans}}
\newcommand{\pr}{\mathrm{pr}}
\newcommand{\Fun}{\mathrm{Fun}}
\newcommand{\colim}{\mathrm{colim}}

% simplicial sets
\newcommand{\DDelta}{\Updelta}
\newcommand{\Sing}{\operatorname{Sing}}

% ideal theory
\newcommand{\mfrak}{\mathfrak{m}}
\newcommand{\afrak}{\mathfrak{a}}
\newcommand{\bfrak}{\mathfrak{b}}
\newcommand{\pfrak}{\mathfrak{p}}
\newcommand{\qfrak}{\mathfrak{q}}

% number theory
\newcommand{\Tr}{\mathrm{Tr}}
\newcommand{\Nm}{\mathrm{Nm}}
\newcommand{\Gal}{\mathrm{Gal}}
\newcommand{\Frob}{\mathrm{Frob}}
\setlength{\headheight}{40pt}


\newenvironment{solution}
  {\renewcommand\qedsymbol{$\blacksquare$}
  \begin{proof}[Solution]}
  {\end{proof}}
\renewcommand\qedsymbol{$\blacksquare$}

\usepackage{amsmath, amssymb, tikz, amsthm, csquotes, multicol, footnote, tablefootnote, biblatex, wrapfig, float, quiver, mathrsfs, cleveref, enumitem, upgreek}
\addbibresource{refs.bib}
\theoremstyle{definition}
\newtheorem{theorem}{Theorem}[section]
\newtheorem{lemma}[theorem]{Lemma}
\newtheorem{corollary}[theorem]{Corollary}
\newtheorem{exercise}[theorem]{Exercise}
\newtheorem{question}[theorem]{Question}
\newtheorem{example}[theorem]{Example}
\newtheorem{proposition}[theorem]{Proposition}
\newtheorem{conjecture}[theorem]{Conjecture}
\newtheorem*{remark}{Remark}
\newtheorem{definition}[theorem]{Definition}
\numberwithin{equation}{section}
\begin{document}
\large
\title[Infinity Categories]{MATH 292Z: First Steps in Infinity Categories}
\author{Wern Juin Gabriel Ong}
\address{Bowdoin College, Brunswick, Maine 04011}
\email{gong@bowdoin.edu}
\urladdr{https://wgabrielong.github.io/}
\maketitle
\section*{Preliminaries}
These notes roughly correspond to the course \textbf{MATH 292Z: First Steps in Infinity Categories} taught by Prof. Michael Hopkins at Harvard University in the Fall 2023 semester. These notes are \LaTeX-ed after the fact with significant alteration and are subject to misinterpretation and mistranscription. Use with caution. Any errors are undoubtedly my own and any virtues that could be ascribed to these notes ought be attributed to the instructor and not the typist. 
\tableofcontents
\section{Lecture 1 -- 6th September 2023}
Understanding infinity categories is a daunting task. Ideally, we will end this class with the student being conversant in the language of infinity categories, though this may be difficult to achieve. Let us start with the definition of a (1-)category. 
\begin{definition}[Category]
    A category $\Csf$ is a class $\Obj(\Csf)$ and for each $A,B\in\Obj(\Csf)$ a set $\Mor_{\Csf}(A,B)$ of morphisms from $A$ to $B$ such that
    \begin{enumerate}[label=(\alph*)]
        \item $\id_{A}\in\Mor_{\Csf}(A,A)$,
        \item A composition law $\Mor_{\Csf}(B,C)\times\Mor_{\Csf}(A,B)\to\Mor_{\Csf}(A,C)$ by $(g,f)\mapsto g\circ f$ that is unital $\id_{A}\circ f=f=f\circ\id_{B}$ and associating $(g\circ f)\circ h=g\circ(f\circ h)$. 
    \end{enumerate}
\end{definition}
\begin{remark}
  We consider the underlying collection of objects of a category a class to avoid set-theoretic difficulties such as Russell's Paradox when dealing with the category of sets. 
\end{remark}
Let us consider some examples. 
\begin{example}
  $\Sets$ whose objects are sets and whose morphisms are set functions. 
\end{example}
\begin{example}
  $\Top$ whose objects are topological spaces and whose morphisms are continuous maps between them. 
\end{example}
\begin{example}[Representable Spaces of Groups]\label{ex:representable spaces}
Let $G$ be a group. We can construct a category $\Csf_{G}$ whose object is a point $*$ and whose morphisms are given by $\Mor_{\Csf_{G}}(*,*)=G$ where the unital element is $\id_{G}$ and composition given by group multiplication.
\end{example}
\begin{example}[Fundemental Groupoid]
Let $X$ be a topological space. Consider $\Pi_{\leq1}X$ the fundamental groupoid of $X$ whose object is $X$ itself and whose morphisms $\Mor_{\Pi_{\leq1}X}(x,y)$ homotopy classes of paths from $x,y\in X$. Note that the unital and associative properties of path composition only hold after passing to homotopy equivalence. 
\end{example}
Let us now define a functor. 
\begin{definition}[Functor]
  Let $\Csf,\Dsf$ be categories. A functor $F:\Csf\to\Dsf$ consists of a function $F:\Obj(\Csf)\to\Obj(\Dsf)$ such that for all $A,B\in\Obj(\Csf)$, there is a function $\Mor_{\Csf}(A,B)\to\Mor_{\Dsf}(F(A),F(B))$ satisfying $F(\id_{A})=\id_{F(A)}$ and $F(g\circ f)=F(g)\circ F(f)$. 
\end{definition}
A functor is a map between categories. Correspondingly, we can define a map between functors as a natural transformation. 
\begin{definition}[Natural Transformation]
Let $F,G:\Csf\to\Dsf$ be functors. $T$ is a natural transformation between $F$ and $G$ consists of a morphism $T(A):F(A)\to G(A)$ in $\Dsf$ for all $A\in\Obj(\Csf)$ and the diagram
$$\begin{tikzcd}
	{F(A)} && {G(A)} \\
	{F(B)} && {G(B)}
	\arrow["{T(A)}", from=1-1, to=1-3]
	\arrow["{T(B)}", from=2-1, to=2-3]
	\arrow["{T(f)}"', from=1-1, to=2-1]
	\arrow["{G(f)}"', from=1-3, to=2-3]
\end{tikzcd}$$
commutes for all $f\in\Mor_{\Csf}(A,B)$. 
\end{definition}
\begin{example}
  Let $X$ be a topological space. We can construct a covering space of $X$ denoted $\widetilde{X}$ such that for the solid diagram, 
  $$% https://q.uiver.app/#q=WzAsNCxbMCwwLCIqIl0sWzAsMSwiWzAsMV0iXSxbMiwxLCJYIl0sWzIsMCwiXFx3aWRldGlsZGV7WH0iXSxbMCwzXSxbMywyLCJwIl0sWzAsMV0sWzEsMl0sWzEsMywiXFxleGlzdHMhIiwxLHsic3R5bGUiOnsiYm9keSI6eyJuYW1lIjoiZGFzaGVkIn19fV1d
  \begin{tikzcd}
    {*} && {\widetilde{X}} \\
    {[0,1]} && X
    \arrow[from=1-1, to=1-3]
    \arrow["p", from=1-3, to=2-3]
    \arrow[from=1-1, to=2-1]
    \arrow[from=2-1, to=2-3]
    \arrow["{\exists!}"{description}, dashed, from=2-1, to=1-3]
  \end{tikzcd}$$
  there is a unique dotted arrow lifting a path in $X$ to $\widetilde{X}$. In other words, this is a functor $\Pi_{\leq 1}X\to\Sets$ by $x\mapsto p^{-1}(x)$. 
\end{example}
In fact, and old theorem of Grothendieck states the following. 
\begin{theorem}[Grothendieck]
  Let $X$ be a topological space. The category of covering spaces of $X$ is equivalent ot the category of functors $\Pi_{\leq 1}X\to\Sets$. 
\end{theorem}
Within this framework, topological statements like Van-Kampen's theorem become much easier. 

In topology, we have paths, and homotopies between paths. Indeed, we can construct homotopies between homotopies, and so on. This is the root of the concept of an infinity category. 

And much of the topological information we wish to incorporate above is easiest done with simplices, leading the formalism of infinity categories to be built on the backbone of the rich combinatorial structure of simplical sets. Let's start with an easy example of the partially ordered set, or poset. 
\begin{definition}[Poset]
  A poset $(S,\leq)$ is a set $S$ together with a relation $\leq$ such that
  \begin{enumerate}[label=(\alph*)]
    \item $a\leq a$ for all $a\in S$, 
    \item if $a\leq b$ and $b\leq c$ then $a\leq c$, 
    \item and if $a\leq b$ and $b\leq a$ then $a=b$. 
  \end{enumerate}
\end{definition}
A poset can be seen as a category whose objects are the elements of $S$ and $$\Mor_{S}(a,b)=\begin{cases}\emptyset & a\nleq b \\ * & a\leq b.\end{cases}$$ Indeed, one can show that any category with at most one morphism between any two objects is a poset under the relation induced by the morphisms. We can now introduce simplical sets. 
\begin{definition}[Simplices]
  The category of simplices $\DDelta$ has objects finite totally ordered sets $[n]=\{0,1,2,\dots,n\}$ and morphisms order-preserving maps. 
\end{definition}
There is a functor $\DDelta\to\Top$ taking $[n]$ to $|\Delta^{n}|$ the standard $n$-simplex, the convex hull of $\{e_{0},\dots,e_{n}\}$ in $\RR^{n+1}$ and taking a morphism $[n]\to[m]$ to linear-maps $e_{i}\mapsto e_{f(i)}$. Such maps $f:[n]\to[m]$ are determined by their (finite) images $0\leq f(1)\leq f(2)\leq\dots\leq f(n)\leq m$ which we will write as $\langle f(1), \dots, f(n)\rangle$. 

\begin{example}
  There are maps $\langle i, i+1\rangle:[1]\to[n]$ for $0\leq i\leq n-1$.  
  $$\begin{tikzcd}
    && 2 \\
    \\
    & 0 && 1 \\
    \\
    1 & 2
    \arrow[from=3-2, to=1-3]
    \arrow[""{name=0, anchor=center, inner sep=0}, from=3-4, to=1-3]
    \arrow[from=3-2, to=3-4]
    \arrow[""{name=1, anchor=center, inner sep=0}, from=5-1, to=5-2]
    \arrow[curve={height=-6pt}, shorten <=9pt, shorten >=9pt, rightarrow, from=1, to=0]
  \end{tikzcd}$$
  The above demonstrates the map $\langle1,2\rangle:[1]\to[2]$. 
\end{example}

Let $X$ be an arbitrary topological space. We can get a contravariant functor $\DDelta\to\Sets$ by taking $[n]\to\Mor_{\Top}(|\Delta^{n}|,X)$. We denote this functor $\Sing X:\DDelta\to\Sets$. This allows us to define simplical sets. 
\begin{definition}[Simplicial Sets]
  The category of simplicial sets $\SSets$ is the functor category whose objects are contravariant functors $\DDelta\to\Sets$ and morphisms natural transformations between such functors. 
\end{definition}

Let us switch directions for a moment and consider nerves of categories. We have already seen how a poset can be made into a category. Analogously, a totally ordered set can be made into a category in a similar way, yielding a functor $\DDelta\to\Cat$, the category whose objects are categories and whose morphisms are functors. Nerves help us understand how simplices live within categories. 
\begin{definition}[Nerve]
  Let $\Csf$ be a category. The nerve of $\Csf$ is the simplicial set $N\Csf$ so that 
  $$(N\Csf)_{n}=\Fun([n],\Csf)$$
  the set of functors from $[n]$ to $\Csf$ such that the simplical operators $f:[n]\to[m]$ act by pre-composition where $a\mapsto a\circ f$ for $a:[n]\to\Csf$ in $(N\Csf)_{n}$. 
\end{definition}
We can make this more concrete with a few examples. 
\begin{example}
  $(N\Csf)_{0}$ is the simplicial set of maps from the point to $\Csf$ and hence corresponds to the objects of $\Csf$, $\Obj(\Csf)$. 
  \\\\
  $(N\Csf)_{1}$ is the simplicial set of maps from $0\to 1$ to $\Csf$ and hence corresponds to the morphisms of $\Csf$. 
  \\\\
  $(N\Csf)_{2}$ is the simplicial set of maps 
  $$\begin{tikzcd}
    0 && 1 \\
    && 2
    \arrow[from=1-1, to=1-3]
    \arrow[from=1-3, to=2-3]
    \arrow[from=1-1, to=2-3]
  \end{tikzcd}$$
  to $\Csf$, that is, to pairs of composable morphisms in $\Csf$. 
  \\\\
  More generally $(N\Csf)_{n}$ is the set of composable $n$-tuples of morphisms in $\Csf$. 
\end{example}
\begin{theorem}
  A simplicial set $X$ is isomorphic to the nerve of some category $\Csf$ if and only if for all $n\geq2$, the map 
  $$X_{n}\to \underbrace{X_{1}\times_{X_{0}}X_{1}\times_{X_{0}}\times_{X_{0}}\dots\times_{X_{0}}X_{1}}_{n\text{ times}}$$
  by
  $$g\mapsto (\langle0,1\rangle^{*}g, \langle1,2\rangle^{*}g,\dots,\langle n-1,n\rangle^{*}g)$$
  is an isomorphism. 
\end{theorem}
This proof follows that in \cite[Proposition 1.4.8]{Rezk}.
\begin{proof}
  $(\Longrightarrow)$ Suppose $X=N\Csf$ for some category $\Csf$. A tuple of $n$ composable morphisms in $\Csf$ is precisely the datum of the $n$-nerve $(N\Csf)_{n}$. 
  \\\\
  $(\Longleftarrow)$ Suppose $X$ is a simplicial set such that the function above is a bijection. We construct the category $\Csf$ whose objects $\Obj(\Csf)=X_{0}$ and whose morphisms are $X_{1}$. For $g\in X_{1}$, we have two functors $X_{1}\to X_{0}$ taking $g\mapsto\langle0\rangle^{*}g$ its source object and $g\to\langle1\rangle^{*}g$ its target object. Since $0\to0$ is a morphism in $|\DDelta^{1}|$, we trivially have identity morphisms on each object. Moreover, given a morphism $f\in X_{1}$, we can recover its source object $\langle0\rangle^{*}f$ and its target object $\langle1\rangle^{*}f$. And for any $(g,h)\in X_{1}\times_{X_{0}}X_{1}$, the composite $h\circ g\in X_{2}$ is the unique morphism where $\langle0,1\rangle^{*}(g\circ h)=h$ and $\langle 1,2\rangle^{*}(g\circ h)=g$ where uniqueness follows from bijectivity. We now claim that for any $g\in X_{n}$ and $0\leq i\leq j\leq k\leq n$ we have $\langle i,k\rangle^{*}g=\langle j,k\rangle^{*}g\circ\langle j,i\rangle^{*}g$ where each of $\langle i,k\rangle^{*}g,\langle j,k\rangle^{*}g,\langle j,i\rangle^{*}g\in X_{1}$ as images of maps $[1]\to[n]$. Indeed any element of $X_{2}$ decomposes in this way. We can thus define maps $X_{n}\to (N\Csf)_{n}$ by sending $g\in X_{n}$ to $[n]\to\Csf$ by $g\mapsto g$ and $(i\to j)\mapsto \langle j, i\rangle^{*}g$ which is a functor by the above. We can verify these maps are bijections since $((i-1)\to i)\mapsto\langle i, i-1\rangle^{*}g$. Furthermore, if $f:[m]\to [n]$ is a map of simplices, we compute
  $$\langle j,i\rangle^{*}(g\circ f) = \langle f(j), f(i)\rangle^{*}g$$
  giving an isomorphism $X\to N\Csf$ of simplical sets. 
\end{proof}
We now discuss the further significance of $\Sing X$. To do this, we first have to discuss horns and Kan complexes. Horns are subcomplexes of standard simplicies. 
\begin{definition}[Horn]
  For each $n\geq1$, there are subcomplexes $\Lambda_{j}^{n}\subseteq|\Delta^{n}|$ for each $0\leq j\leq n$ defined by 
  $$(\Lambda_{j}^{n})_{k}=\{f:[k]\to[n]|([n]\setminus[j])\not\subseteq f([k])\}.$$
\end{definition}
Morally, we should think of a horn $\Lambda^{n}_{j}$ is the union of faces of the $n$ simplex other than the $j$th. 
\begin{example}[1-Horns]
  The horns in $|\Delta^{1}|$ are the vertices $\Lambda^{1}_{0}=\{0\}, \Lambda^{1}_{1}=\{1\}$. 
\end{example}
\begin{example}[2-Horns]
  There are three horns in the 2-simplex. 
  $$% https://q.uiver.app/#q=WzAsMTMsWzAsMiwiMCJdLFsxLDAsIjIiXSxbMiwyLCIxIl0sWzQsMiwiMCJdLFs2LDBdLFs1LDAsIjIiXSxbNiwyLCIxIl0sWzgsMiwiMCJdLFsxMCwyLCIxIl0sWzksMCwiMiJdLFs5LDMsIlxcTGFtYmRhXnsyfV97Mn0iXSxbNSwzLCJcXExhbWJkYV57Mn1fezF9Il0sWzEsMywiXFxMYW1iZGFeezJ9X3swfSJdLFswLDFdLFswLDJdLFszLDZdLFs2LDVdLFs3LDldLFs4LDldXQ==
  \begin{tikzcd}
    & 2 &&&& 2 & {} &&& 2 \\
    \\
    0 && 1 && 0 && 1 && 0 && 1 \\
    & {\Lambda^{2}_{0}} &&&& {\Lambda^{2}_{1}} &&&& {\Lambda^{2}_{2}}
    \arrow[from=3-1, to=1-2]
    \arrow[from=3-1, to=3-3]
    \arrow[from=3-5, to=3-7]
    \arrow[from=3-7, to=1-6]
    \arrow[from=3-9, to=1-10]
    \arrow[from=3-11, to=1-10]
  \end{tikzcd}$$
\end{example}
We can now define Kan complexes. 
\begin{definition}[Kan Complex]
  A simplicial set is a Kan complex if for all $k,n$ the solid diagram
  $$% https://q.uiver.app/#q=WzAsNCxbMCwwLCJcXExhbWJkYV57bn1fe2t9Il0sWzIsMCwiWCJdLFswLDEsInxcXERlbHRhXntufXwiXSxbMSwwXSxbMCwxXSxbMiwxLCIiLDAseyJzdHlsZSI6eyJib2R5Ijp7Im5hbWUiOiJkYXNoZWQifX19XSxbMCwyLCIiLDAseyJzdHlsZSI6eyJ0YWlsIjp7Im5hbWUiOiJob29rIiwic2lkZSI6InRvcCJ9fX1dXQ==
  \begin{tikzcd}
    {\Lambda^{n}_{k}} & {} & X \\
    {|\Delta^{n}|}
    \arrow[from=1-1, to=1-3]
    \arrow[dashed, from=2-1, to=1-3]
    \arrow[hook, from=1-1, to=2-1]
  \end{tikzcd}$$
  admits a dotted map making the diagram commute. 
\end{definition}
A theorem of Kan states the following:
\begin{theorem}[Kan]
  $\Sing X$ is a Kan complex. 
\end{theorem}
This eventually led to the definition of quasicategories. 
\begin{definition}[Quasicategory; Boardman-Vogt, Joyal, Lurie]\label{def:quasicategory}
  A simplicial set is a quasicategory if every solid diagram
  $$% https://q.uiver.app/#q=WzAsNCxbMCwwLCJcXExhbWJkYV57bn1fe2t9Il0sWzIsMCwiWCJdLFswLDEsInxcXERlbHRhXntufXwiXSxbMSwwXSxbMCwxXSxbMiwxLCJcXGV4aXN0cyEiLDIseyJzdHlsZSI6eyJib2R5Ijp7Im5hbWUiOiJkYXNoZWQifX19XSxbMCwyLCIiLDAseyJzdHlsZSI6eyJ0YWlsIjp7Im5hbWUiOiJob29rIiwic2lkZSI6InRvcCJ9fX1dXQ==
  \begin{tikzcd}
    {\Lambda^{n}_{k}} & {} & X \\
    {|\Delta^{n}|}
    \arrow[from=1-1, to=1-3]
    \arrow["{\exists!}"', dashed, from=2-1, to=1-3]
    \arrow[hook, from=1-1, to=2-1]
  \end{tikzcd}$$
  admits a dotted map making the diagram commute. 
\end{definition}
We will revisit these ideas in the following lecture.  
\section{Lecture 2 -- 11th September 2023}
We made a very quick pass of simplicial sets in the previous lecture. Let us go through it in more detail. 

Let $\DDelta$ be the category whose objects are finite ordered sets and whose morphisms are order-preserving maps. Following the notation in Rezk's text \cite{Rezk}, for a morphism $[n]\to[k]$, we write it $\langle k_{0},\dots,k_{n}\rangle$ where $0\leq k_{0}\leq\dots\leq k_{n}\leq k$. In the category $\DDelta$, there are two types of distinguished maps $d_{i}$ and $s_{i}$ which we now define. 
\begin{definition}[Face Maps]\label{def:face maps}
  In the category $\DDelta$, there are maps $d^{k}:[n-1]\to[n]$ by $\langle 0,1,\dots,\hat{k},\dots,n\rangle=\langle 0,1,\dots,k-1,k+1,\dots,n\rangle\to[n]$. 
\end{definition}
\begin{example}
  We should think of these as the inclusion of a face into the simplex. 
  $$% https://q.uiver.app/#q=WzAsNyxbMiwyLCIxIl0sWzQsMiwiMCJdLFs2LDMsIjEiXSxbNiwxLCIyIl0sWzUsMCwiMyJdLFswLDEsIjAiXSxbMiwwLCIyIl0sWzIsNF0sWzEsMl0sWzIsM10sWzEsNF0sWzEsM10sWzMsNF0sWzUsMF0sWzAsNl0sWzUsNl0sWzE0LDcsImReezJ9IiwwLHsiY3VydmUiOjEsInNob3J0ZW4iOnsic291cmNlIjoyMCwidGFyZ2V0IjoyMH19XV0=
  \begin{tikzcd}
    && 2 &&& 3 \\
    0 &&&&&& 2 \\
    && 1 && 0 \\
    &&&&&& 1
    \arrow[""{name=0, anchor=center, inner sep=0}, from=4-7, to=1-6]
    \arrow[from=3-5, to=4-7]
    \arrow[from=4-7, to=2-7]
    \arrow[from=3-5, to=1-6]
    \arrow[from=3-5, to=2-7]
    \arrow[from=2-7, to=1-6]
    \arrow[from=2-1, to=3-3]
    \arrow[""{name=1, anchor=center, inner sep=0}, from=3-3, to=1-3]
    \arrow[from=2-1, to=1-3]
    \arrow["{d^{2}}", curve={height=6pt}, shorten <=23pt, shorten >=23pt, rightarrow, from=1, to=0]
  \end{tikzcd}$$
  Here, we show the inclusion $d^{2}:[2]\to[3]$ by $\langle0,1,3\rangle$, taking the 2-simplex to the face ``opposite'' the vertex 2, that is, the face bounded by the vertices 0, 1, and 3. 
\end{example}
We can also define degeneracy maps as follows. 
\begin{definition}[Degeneracy Map]\label{def:degeneracy map}
  In the category $\DDelta$, there are maps $s^{k}:[n]\to[n-1]$ by $[n]\mapsto\langle 0,1,2,\dots,k,k,k+1,\dots,n\rangle$. 
\end{definition}
\begin{example}
  We should think of this as collapsing the $k$th to the $(k-1)$th vertex. 
  $$% https://q.uiver.app/#q=WzAsNixbMCwwLCIxIl0sWzAsMV0sWzAsMiwiMCJdLFsyLDEsIjIiXSxbMCw0LCIwIl0sWzIsNCwiMSJdLFswLDNdLFsyLDNdLFsyLDBdLFs0LDVdLFs3LDksInNeezB9IiwwLHsiY3VydmUiOi0xLCJzaG9ydGVuIjp7InNvdXJjZSI6MTAsInRhcmdldCI6MTB9fV1d
  \begin{tikzcd}
    1 \\
    {} && 2 \\
    0 \\
    \\
    0 && 1
    \arrow[from=1-1, to=2-3]
    \arrow[""{name=0, anchor=center, inner sep=0}, from=3-1, to=2-3]
    \arrow[from=3-1, to=1-1]
    \arrow[""{name=1, anchor=center, inner sep=0}, from=5-1, to=5-3]
    \arrow["{s^{0}}", curve={height=-6pt}, shorten <=5pt, shorten >=5pt, rightarrow, from=0, to=1]
  \end{tikzcd}$$
  Here we show $s^{0}:[2]\to[1]$ by $\langle 0,0,1\rangle$ collapsing the vertex 1 to the vertex 0. 
\end{example}
With these operations on $\DDelta$, one can in fact show the following theorem. 
\begin{theorem}
  If $f$ is a morphism in $\DDelta$, then $f$ factors as the composition of face and degeneracy maps. 
\end{theorem}
We omit the proof. 

Now recall that a simplicial set $X$ is a covariant functor $X:\DDelta^{\Opp}\to\Sets$. We write $$X_{n}=\Fun([n], X)
.$$ For a simplicial set $X$, we have a diagram
$$% https://q.uiver.app/#q=WzAsNSxbMCwwLCJYX3swfSJdLFsyLDAsIlhfezF9Il0sWzQsMCwiWF97Mn0iXSxbNSwwLCJcXGRvdHMiXSxbNiwwLCJcXGRvdHMiXSxbMCwxLCJzXnswfSIsMV0sWzEsMCwiZF57MX0iLDEseyJvZmZzZXQiOi0zLCJjdXJ2ZSI6LTF9XSxbMCwxLCJkXnswfSIsMSx7Im9mZnNldCI6LTMsImN1cnZlIjotMSwic3R5bGUiOnsidGFpbCI6eyJuYW1lIjoiYXJyb3doZWFkIn0sImhlYWQiOnsibmFtZSI6Im5vbmUifX19XSxbMSwyLCJzXnsxfSIsMSx7Im9mZnNldCI6MywiY3VydmUiOjF9XSxbMiwxLCJkXnsxfSIsMV0sWzEsMiwic157MH0iLDEseyJvZmZzZXQiOi0yLCJjdXJ2ZSI6LTF9XSxbMiwxLCJkXnsyfSIsMSx7Im9mZnNldCI6LTUsImN1cnZlIjotMn1dLFsyLDEsImReezB9IiwxLHsib2Zmc2V0Ijo1LCJjdXJ2ZSI6Mn1dXQ==
\begin{tikzcd}
	{X_{0}} && {X_{1}} && {X_{2}} & \dots & \dots
	\arrow["{s^{0}}"{description}, from=1-1, to=1-3]
	\arrow["{d^{1}}"{description}, shift left=3, curve={height=-6pt}, from=1-3, to=1-1]
	\arrow["{d^{0}}"{description}, shift left=3, curve={height=-6pt}, tail reversed, no head, from=1-1, to=1-3]
	\arrow["{s^{1}}"{description}, shift right=3, curve={height=6pt}, from=1-3, to=1-5]
	\arrow["{d^{1}}"{description}, from=1-5, to=1-3]
	\arrow["{s^{0}}"{description}, shift left=2, curve={height=-6pt}, from=1-3, to=1-5]
	\arrow["{d^{2}}"{description}, shift left=5, curve={height=-12pt}, from=1-5, to=1-3]
	\arrow["{d^{0}}"{description}, shift right=5, curve={height=12pt}, from=1-5, to=1-3]
\end{tikzcd}$$
induced by the degeneracy and face maps. 
\begin{definition}[$n$-Simplex]
  The $n$-simplex is the functor $\Delta^{n}:\DDelta^{\Opp}\to\Sets$ by $[k]\mapsto\Mor_{\DDelta}([k],[n])$. 
\end{definition}
\begin{remark}
  Note that $\Delta^{n}$ is the functor $\DDelta^{\Opp}\to\Sets$, while the topological $n$-simplex $|\Delta^{n}|\simeq S^{n}\in\Obj(\Top)$ is topological space homeomorphic (and homotopic) to the $n$-sphere. 
\end{remark}
In this way, we can think of $\Delta^{n}$ as the functor represented by $[n]\in\Obj(\DDelta)$. More generally, one can define representable functors as follows. 
\begin{definition}[Representable Functor]
  Let $\Csf$ be a category. A functor $F:\Csf\to\Sets$ is representable if there exists $A\in\Obj(\Csf)$ such that there exists an isomorphism of functors $F\to\Mor_{\Csf}(-,A)$. 
\end{definition}
Generally, consider $F:\Csf^{\Opp}\to\Sets$ a contravariant functor from an arbitrary category $\Csf$ to $\Sets$. Let $f:B\to A$. $F(f):F(B)\to F(A)$ is a map betewen sets. For $u\in F(A)$, $u$ determines a natural transformation of functors $\NatTrans(\Mor_{\Csf}(-,A),F)$. This is Yoneda's lemma. We adapt the version from \cite[Theorem 2.2.4]{Riehl}.  
\begin{lemma}[Yoneda]\label{lem:yoneda}
  For a contravariant functor $F:\Csf\to\Sets$ and $A\in\Obj(\Csf)$, there is a bijection 
  $$\NatTrans(\Mor_{\Csf}(-,A), F)\to F(A)$$
  that associates a natural transformation of functors $\alpha:\Mor_{\Csf}(-,A)\Rightarrow F$ to the element $\alpha(\id_{A})\in F(A)$, natural in both $A$ and $F$. 
\end{lemma}
Let us recall the definition of quasicategories as in \Cref{def:quasicategory}. 
\begin{definition}[Quasicategory]
  A simplcial set $X$ is a quasicategory if every inner horn $\Lambda^{n}_{k}$ where $0<k<n$ has a fill. 
\end{definition}
In other words, for every solid diagram, 
$$% https://q.uiver.app/#q=WzAsMyxbMCwwLCJcXExhbWJkYV57bn1fe2t9Il0sWzAsMSwiXFxEZWx0YV57bn0iXSxbMiwwLCJYIl0sWzAsMSwiIiwxLHsic3R5bGUiOnsidGFpbCI6eyJuYW1lIjoiaG9vayIsInNpZGUiOiJib3R0b20ifX19XSxbMCwyXSxbMSwyLCJcXGV4aXN0cyIsMSx7InN0eWxlIjp7ImJvZHkiOnsibmFtZSI6ImRhc2hlZCJ9fX1dXQ==
\begin{tikzcd}
	{\Lambda^{n}_{k}} && X \\
	{\Delta^{n}}
	\arrow[hook', from=1-1, to=2-1]
	\arrow[from=1-1, to=1-3]
	\arrow["\exists"{description}, dashed, from=2-1, to=1-3]
\end{tikzcd}$$
there is a dotted map making the diagram commute. 

Let us now fix some notation that we will use going forward. For $X$ a simplicial set we denote $X_{n}=\Fun([n],X)$ and $\alpha\in\Mor_{\DDelta}([k],[n])$, we have $X_{\alpha}:X_{n}\to X_{k}$. 
\begin{theorem}\label{thm: nerve iff unique filler}
  A simplicial set is the nerve of some category if and only if every inner horn has a unique filler. 
\end{theorem}
In other words, for every solid diagram, 
$$% https://q.uiver.app/#q=WzAsMyxbMCwwLCJcXExhbWJkYV57bn1fe2t9Il0sWzAsMSwiXFxEZWx0YV57bn0iXSxbMiwwLCJYIl0sWzAsMSwiIiwxLHsic3R5bGUiOnsidGFpbCI6eyJuYW1lIjoiaG9vayIsInNpZGUiOiJib3R0b20ifX19XSxbMCwyXSxbMSwyLCJcXGV4aXN0cyEiLDEseyJzdHlsZSI6eyJib2R5Ijp7Im5hbWUiOiJkYXNoZWQifX19XV0=
\begin{tikzcd}
	{\Lambda^{n}_{k}} && X \\
	{\Delta^{n}}
	\arrow[hook', from=1-1, to=2-1]
	\arrow[from=1-1, to=1-3]
	\arrow["{\exists!}"{description}, dashed, from=2-1, to=1-3]
\end{tikzcd}$$
there is a dotted map making the diagram commute. 

We omit the proof of \Cref{thm: nerve iff unique filler} which can be found in \cite[\S 1.7.10]{Rezk}. 

We want to consider an analogue of \Cref{thm: nerve iff unique filler} in the case of quasicategories. Let $X$ be a quasicategory with objects $X_{0}$ and morphisms $X_{1}$. For $f\in X_{0}$, let $\langle0\rangle^{*}f$ denote the source of the map $f$ and $\langle1\rangle^{*}f$ its target. Let 
$$% https://q.uiver.app/#q=WzAsMyxbMCwwLCJBIl0sWzAsMSwiQiJdLFsyLDAsIkMiXSxbMSwyLCJnIiwyXSxbMCwxLCJmIiwyXV0=
\begin{tikzcd}
	A && C \\
	B
	\arrow["g"', from=2-1, to=1-3]
	\arrow["f"', from=1-1, to=2-1]
\end{tikzcd}$$
be the image of a horn $\Lambda^{2}_{1}$ and $h=\tau\langle0,2\rangle:A\to C$ be the fill. 
$$% https://q.uiver.app/#q=WzAsMyxbMCwwLCJBIl0sWzAsMSwiQiJdLFsyLDAsIkMiXSxbMSwyLCJnIiwyXSxbMCwxLCJmIiwyXSxbMCwyLCJcXHRhdVxcbGFuZ2xlMCwyXFxyYW5nbGU9aCJdXQ==
\begin{tikzcd}
	A && C \\
	B
	\arrow["g"', from=2-1, to=1-3]
	\arrow["f"', from=1-1, to=2-1]
	\arrow["{\tau\langle0,2\rangle=h}", from=1-1, to=1-3]
\end{tikzcd}$$
Since the fill is not unique, we say $\tau$ witnesses $h$ as the composition of $g$ with $f$. This begs the question of how we can perform some operation to make $X$ into a category. This is done via the construction of the fundamental category, which can be done on simplicial sets in general. 
\begin{definition}[Fundamental Category]
  Let $X$ be a simplicial set. A category $\Csf$ is the fundamental category of $X$ if every map $X\to\Dsf$ for $\Dsf$ factors through $\Csf$ and $\Csf$ is final with respect to that property. 
\end{definition}
Fundamental categories are in general hard to construct, but much easier in the case of quasicategories. Indeed if $X$ is a quasicategory, its fundamental category coincides with its homotopy category $hX$. The homotopy category will have objects $X_{0}$ and morphisms those in $X_{1}$ up to ``witnessing''-equivalence. We will have to discuss these notions of equivalence before defining the homotopy category rigorously. 
\begin{definition}[Left-Equivalence]\label{def:left equivalence}
  Let $f,g\in\Mor_{X}(A,B)$. We say that $f$ is left-equivalent to $g$ if there is $\tau\in X_{2}$ such that $\tau\langle0,1\rangle=\id_{A}, \tau\langle0,2\rangle=f, \tau\langle1,2\rangle=g$. 
  $$% https://q.uiver.app/#q=WzAsMyxbMCwwLCJBIl0sWzAsMSwiQSJdLFsyLDEsIkIiXSxbMCwyLCJmIiwxXSxbMSwyLCJnIiwxXSxbMCwxLCJcXGlkX3tBfSIsMV1d
  \begin{tikzcd}
    A \\
    A && B
    \arrow["f"{description}, from=1-1, to=2-3]
    \arrow["g"{description}, from=2-1, to=2-3]
    \arrow["{\id_{A}}"{description}, from=1-1, to=2-1]
  \end{tikzcd}$$
\end{definition}
Similarly, we have right-equivalence. 
\begin{definition}[Right-Equivalence]\label{def:right equivalence}
  Let $f,g\in\Mor_{X}(A,B)$. We say that $f$ is right-equivalent to $g$ if there is $\tau\in X_{2}$ such that $\tau\langle1,2\rangle=\id_{B}, \tau\langle0,1\rangle=f, \tau\langle0,2\rangle=g$. 
  $$% https://q.uiver.app/#q=WzAsMyxbMCwxLCJBIl0sWzIsMSwiQiJdLFsyLDAsIkIiXSxbMCwyLCJnIiwxXSxbMSwyLCJcXGlkX3tCfSIsMV0sWzAsMSwiZiIsMV1d
  \begin{tikzcd}
    && B \\
    A && B
    \arrow["g"{description}, from=2-1, to=1-3]
    \arrow["{\id_{B}}"{description}, from=2-3, to=1-3]
    \arrow["f"{description}, from=2-1, to=2-3]
  \end{tikzcd}$$
\end{definition}
One then shows the following proposition before defining the homotopy category. 
\begin{proposition}\label{prop: left eq right equivalence}
  The relations of left equivalence (\Cref{def:left equivalence}) and right equivalence (\Cref{def:right equivalence}) coincide. 
\end{proposition}
We can now define the homotopy category. 
\begin{definition}[Homotopy Category]\label{def:homotopy category}
  Let $X$ be a simplicial set. The homotopy category $hX$ has objects those in $X_{0}$ and morphisms those in $X_{1}$ up to equivalence. 
\end{definition}
\section{Lecture 3 -- 13th September 2023}
We will discuss some constructions in ordinary category theory. Thinking about these constructions will give us a better idea of analogous constructions in infinity category theory. 
\begin{definition}[Isomorphism]\label{def:isomorphism}
  Let $\Csf$ be a category. A morphism $f\in\Mor_{\Csf}(A,B)$ is an isomorphism if there exists $g\in\Mor_{\Csf}(B,A)$ such that $g\circ f=\id_{A}$ and $f\circ g=\id_{B}$.
\end{definition}
One can easily show that isomorphisms are preserved by functors. 
\begin{proposition}
  Let $F:\Csf\to\Dsf$ be a functor. If $f\in\Mor_{\Csf}(A,B)$ is an isomorphism then $F(f)\in\Mor_{\Dsf}(F(A),F(B))$ is an isomorphism. 
\end{proposition}
\begin{proof}
  The statement follows by the commutativity of the following diagram. 
  $$% https://q.uiver.app/#q=WzAsNCxbMCwwLCJBIl0sWzIsMCwiQiJdLFswLDEsIkYoQSkiXSxbMiwxLCJGKEIpIl0sWzEsMywiRiIsMl0sWzAsMiwiRiIsMl0sWzAsMSwiZiIsMCx7Im9mZnNldCI6LTF9XSxbMSwwLCJnIiwwLHsib2Zmc2V0IjotMX1dLFsyLDMsIkYoZikiLDAseyJvZmZzZXQiOi0xfV0sWzMsMiwiRihnKSIsMCx7Im9mZnNldCI6LTF9XV0=
  \begin{tikzcd}
    A && B \\
    {F(A)} && {F(B)}
    \arrow["F"', from=1-3, to=2-3]
    \arrow["F"', from=1-1, to=2-1]
    \arrow["f", shift left, from=1-1, to=1-3]
    \arrow["g", shift left, from=1-3, to=1-1]
    \arrow["{F(f)}", shift left, from=2-1, to=2-3]
    \arrow["{F(g)}", shift left, from=2-3, to=2-1]
  \end{tikzcd}$$
  More explicitly, there exists $g\in\Mor_{\Csf}(B,A)$ such that $g\circ f=\id_{A}$ and $f\circ g=\id_{B}$ so by functoriality we have $F(g)\circ F(f)=\id_{F(A)},F(f)\circ F(g)=\id_{F(B)}$ as desired. 
\end{proof}
The acute would percieve that \Cref{def:isomorphism} may be slightly problematic since it involves some choice of $g\in\Mor_{\Csf}(B,A)$. The following proposition resolves this issue, showing that such a choice is unique. 
\begin{proposition}
  Let $\Csf$ be a category and $f\in\Mor_{\Csf}(A,B)$ is an isomorphism. If $g_{1},g_{2}\in\Mor_{\Csf}(B,A)$ satisfies $g_{i}\circ f=\id_{A}$ and $f\circ g_{i}=\id_{B}$ for $i\in\{1,2\}$ then $g_{1}=g_{2}$. 
\end{proposition}
\begin{proof}
  We have $g_{1}=g_{1}\circ\id_{Y}=g_{1}\circ f\circ g_{2}=\id_{X}\circ g_{2}=g_{2}$. 
\end{proof}
In some settings, it can be difficult to find such $g$ that satisfies the composition identities as set out in \Cref{def:isomorphism}. Fortunately, we can use the Yoneda functors from \Cref{lem:yoneda} to characterize isomorphisms. 
\begin{theorem}[Representable Functors Determine Isomorphism]\label{thm:rep functors determine iso}
  Let $\Csf$ be a category. A map $f:A\to B$ is an isomorphism if and only if there is an isomorphic natural transformation of the functors they represent $\Mor_{\Csf}(-,A)\to\Mor_{\Csf}(-,B)$.
\end{theorem}
\begin{remark}
  It is equivalent to require that for all $Z\in\Obj(\Csf)$, $\Mor_{\Csf}(Z,A)=\Mor_{\Csf}(Z,B)$. 
\end{remark}
\begin{proof}[Proof of \Cref{thm:rep functors determine iso}]
  This follows from the Yoneda lemma, \Cref{lem:yoneda}.  
\end{proof}
Note the increasing levels of abstraction we have encountered. We started with the ``traditional'' definition of isomorphism that one encounters in say a course on the theory of groups, followed by the characterization of isomorphisms via representable functors. Let us look at one more characterization of isomorphisms in the flavor of how we defined the objects of a category as the 0th nerve $(N\Csf)_{0}$, we can define a category $I_{\cong}$ with two objects and two non-identity morphisms
$$% https://q.uiver.app/#q=WzAsMixbMCwwLCIqIl0sWzIsMCwiKiJdLFswLDEsIiIsMCx7Im9mZnNldCI6LTF9XSxbMSwwLCIiLDAseyJvZmZzZXQiOi0xfV1d
\begin{tikzcd}
	{*} && {*}
	\arrow[shift left, from=1-1, to=1-3]
	\arrow[shift left, from=1-3, to=1-1]
\end{tikzcd}$$
and define an isomorphism as follows:
\begin{definition}[Isomorphism]
  Let $\Csf$ be a category. An isomorphism in $\Csf$ is a functor $I_{\cong}\to\Csf$. 
\end{definition}
We will no doubt revisit isomorphisms in the infinity categorical setting later in the course. 

Now, let us recall the following definitions. 
\begin{definition}[Initial Object]\label{def:initial object}
  Let $\Csf$ be a category. An object $X\in\Obj(\Csf)$ is initial if for all $Y\in\Obj(\Csf)$, there is a unique map $X\to Y$. 
\end{definition}
\begin{definition}[Final Object]\label{def:final object}
  Let $\Csf$ be a category. An object $X\in\Obj(\Csf)$ is final if for all $Y\in\Obj(\Csf)$, there is a unique map $Y\to X$. 
\end{definition}
We now show that final objects (and initial objects, by taking the opposite category) are unique up to unique isomorphism. In other words, they satisfy a universal property. Universal properties are ubiquitous in many fields of mathematics and significantly streamlines one's thinking about mathematical constructions. 
\begin{proposition}
  If $X_{1},X_{2}$ be final objects of a category $\Csf$ then $X_{1}$ is isomorphic to $X_{2}$. 
\end{proposition}
\begin{proof}
  Both $X_{1}$ and $X_{2}$ represent the same functor $\Csf\to\Sets$, taking any object of $\Csf$ to the one-point set $\{*\}$. 
\end{proof}
We have already seen how isomorphisms work in a category. We now define them in the setting of quasicategories. 
\begin{definition}[Isomorphism in Quasicategories]
  Let $X$ be a quasicategory. A morphism $f\in X_{1}$ is an isomorphism if and only if $[f]\in(hX)_{1}$ is an isomorphism. 
\end{definition}
Isomorphisms arising in homotopy categories will soon provide a tool for characterizing Kan complexes. We begin by defining the groupoid. 
\begin{definition}[Groupoid]\label{def:groupoid}
  A category $\Csf$ is a groupoid if all morphisms in $\Csf$ are isomorphisms. 
\end{definition}
\begin{example}
  The fundamental groupoid of a topological space $\Pi_{\leq1}X$ is a groupoid. Morphisms correspond to paths, which are invertible by sending $\gamma(t)\to\gamma(1-t)$. 
\end{example}
The following theorem outlines a correspondence between quasicategories and Kan complexes with the language of groupoids. 
\begin{theorem}\label{thm:kan complex iff hX groupoid}
  A quasicategory $X$ is a Kan complex if and only if its homotopy category $hX$ is a groupoid. 
\end{theorem}
\begin{proof}[Partial Proof of \Cref{thm:kan complex iff hX groupoid}]
  $(\Longrightarrow)$ We prove that if $X$ is a Kan complex, its homotopy category is a groupoid.  
  $$% https://q.uiver.app/#q=WzAsMyxbMCwxLCJBIl0sWzIsMSwiQiJdLFswLDAsIkEiXSxbMCwxLCJmIiwyXSxbMSwyLCJcXGV4aXN0cyEiLDIseyJzdHlsZSI6eyJib2R5Ijp7Im5hbWUiOiJkYXNoZWQifX19XSxbMCwyLCJcXGlkX3tBfSJdXQ==
  \begin{tikzcd}
    A \\
    A && B
    \arrow["f"', from=2-1, to=2-3]
    \arrow["{\exists}"', dashed, from=2-3, to=1-1]
    \arrow["{\id_{A}}", from=2-1, to=1-1]
  \end{tikzcd}$$
  Let $f:A\to B$ be a morphism in $X$ and since $X$ is a Kan complex, the horn $\Lambda^{2}_{0}$ admits a fill by some $\tau\in X_{2}$. One verifies that for $g=\tau\langle1,2\rangle$ we have $g\circ f=\id_{A}$. To show $f\circ g=\id_{B}$ we construct the diagram 
  $$% https://q.uiver.app/#q=WzAsMyxbMCwxLCJBIl0sWzIsMSwiQiJdLFsyLDAsIkIiXSxbMSwyLCJcXGlkX3tCfSIsMl0sWzAsMiwiZiJdLFsxLDAsIlxcZXhpc3RzIiwwLHsic3R5bGUiOnsiYm9keSI6eyJuYW1lIjoiZGFzaGVkIn19fV1d
  \begin{tikzcd}
    && B \\
    A && B
    \arrow["{\id_{B}}"', from=2-3, to=1-3]
    \arrow["f", from=2-1, to=1-3]
    \arrow["\exists", dashed, from=2-3, to=2-1]
  \end{tikzcd}$$
  and by the fact that $X$ is a Kan complex, the horn $\Lambda^{2}_{2}$ admits a fill by some $\tau'\in X_{2}$ such that $\tau'\langle0,1\rangle=h$ satisfying $f\circ h=\id_{B}$. But we have $g=g\circ \id_{B}=g\circ f\circ h=\id_{A}\circ h=h$ proving that there is a unique inverse in the homotopy category. 
\end{proof}
The converse, showing that a homotopy category in which all morphisms are isomorphisms arises as the homotopy category of a Kan complex is much more difficult requiring a significant combinatorial assertion. We will return to Joyal's proof of the converse later in this course. 
\begin{definition}[Quasigroupoid]
  A quasicategory $X$ is a quasigroupoid if $hX$ is a groupoid. 
\end{definition}
As \Cref{thm:kan complex iff hX groupoid} shows, Kan complexes are examples of quasigroupoids. 

\begin{definition}[Small Category]
  A category $\Csf$ is a small category if $\Obj(\Csf)$ is a set as opposed to a class. 
\end{definition}
Having defined a small category, we can define the category of small categories. 
\begin{definition}[Category of Small Categories]
  Denote $\Cat_{1}$ the category of small categories whose objects are small categories and whose morphisms are functors betweeen them. 
\end{definition}
Analogously to nerves, we can consider $(\Cat_{1})_{n}$ be the data of a category $\Csf_{i}$ for each $1\leq i\leq n$, a functor $F_{ij}:\Csf_{i}\to\Csf_{j}$ for each $i\leq j$, and a natural isomorphism of functors $\varepsilon_{ijk}:F_{ik}\Rightarrow F_{jk}\circ F_{ij}$ making the following diagram commute. 
$$% https://q.uiver.app/#q=WzAsNCxbMCwxLCJcXENzZl97aX0iXSxbMywxXSxbNCwxLCJcXENzZl97a30iXSxbMiwwLCJcXENzZl97an0iXSxbMCwzLCJGX3tpan0iXSxbMywyLCJGX3tqa30iXSxbMCwyLCJGX3tpa30iLDJdLFs2LDMsIlxcdmFyZXBzaWxvbl97aWprfSIsMCx7Im9mZnNldCI6LTEsInNob3J0ZW4iOnsic291cmNlIjoxMCwidGFyZ2V0IjozMH19XV0=
\begin{tikzcd}
	&& {\Csf_{j}} \\
	{\Csf_{i}} &&& {} & {\Csf_{k}}
	\arrow["{F_{ij}}", from=2-1, to=1-3]
	\arrow["{F_{jk}}", from=1-3, to=2-5]
	\arrow[""{name=0, anchor=center, inner sep=0}, "{F_{ik}}"', from=2-1, to=2-5]
	\arrow["{\varepsilon_{ijk}}", shift left, shorten <=2pt, shorten >=5pt, Rightarrow, from=0, to=1-3]
\end{tikzcd}$$
Additionally, we require the functors to satisfy the following ``cocycle conditons'' which we now explain. Given $(\Cat_{1})_{3}$ 
$$% https://q.uiver.app/#q=WzAsNCxbMCwzLCJcXENzZl97aX0iXSxbMywyLCJcXENzZl97bH0iXSxbMiwwLCJcXENzZl97a30iXSxbMyw0LCJcXENzZl97an0iXSxbMiwxLCJGX3trbH0iXSxbMCwxLCJGX3tpbH0iLDFdLFszLDIsIkZfe2prfSIsMV0sWzAsMywiRl97aWp9IiwyXSxbMywxLCJGX3tqbH0iLDFdLFswLDIsIkZfe2lrfSIsMV1d
\begin{tikzcd}
	&& {\Csf_{k}} \\
	\\
	&&& {\Csf_{l}} \\
	{\Csf_{i}} \\
	&&& {\Csf_{j}}
	\arrow["{F_{kl}}", from=1-3, to=3-4]
	\arrow["{F_{il}}"{description}, from=4-1, to=3-4]
	\arrow["{F_{jk}}"{description}, from=5-4, to=1-3]
	\arrow["{F_{ij}}"', from=4-1, to=5-4]
	\arrow["{F_{jl}}"{description}, from=5-4, to=3-4]
	\arrow["{F_{ik}}"{description}, from=4-1, to=1-3]
\end{tikzcd}$$
we can ``flatten'' the tetrahedron to observe 
$$% https://q.uiver.app/#q=WzAsOSxbMCwyLCJcXENzZl97aX0iXSxbMiwzLCJcXENzZl97an0iXSxbMiwwXSxbMiwxLCJcXENzZl97bH0iXSxbNCwyLCJcXENzZl97a30iXSxbNywyLCJcXENzZl97aX0iXSxbMTEsMiwiXFxDc2Zfe2t9Il0sWzksMSwiXFxDc2Zfe2x9Il0sWzksMywiXFxDc2Zfe2p9Il0sWzEsMywiRl97amx9IiwxXSxbMCwzLCJGX3tpbH0iXSxbNCwzLCJGX3trbH0iLDJdLFswLDEsIkZfe2lqfSIsMl0sWzEsNCwiRl97amt9IiwyXSxbNSw2LCJGX3tpa30iLDFdLFs1LDcsIkZfe2lsfSJdLFs2LDcsIkZfe2tsfSIsMl0sWzUsOCwiRl97aWp9IiwyXSxbOCw2LCJGX3tqa30iLDJdLFsxMCwxLCJcXHZhcmVwc2lsb25fe2lrbH0iLDIseyJzaG9ydGVuIjp7InNvdXJjZSI6MjAsInRhcmdldCI6NDB9fV0sWzksMTMsIlxcdmFyZXBzaWxvbl97amtsfSIsMCx7Im9mZnNldCI6LTQsInNob3J0ZW4iOnsic291cmNlIjoyMCwidGFyZ2V0IjoyMH19XSxbMTUsMTQsIlxcdmFyZXBzaWxvbl97aWxrfSIsMCx7InNob3J0ZW4iOnsic291cmNlIjoyMCwidGFyZ2V0IjozMH19XSxbMTQsOCwiXFx2YXJlcHNpbG9uX3tpamt9IiwwLHsic2hvcnRlbiI6eyJzb3VyY2UiOjMwLCJ0YXJnZXQiOjIwfX1dXQ==
\begin{tikzcd}
	&& {} \\
	&& {\Csf_{l}} &&&&&&& {\Csf_{l}} \\
	{\Csf_{i}} &&&& {\Csf_{k}} &&& {\Csf_{i}} &&&& {\Csf_{k}} \\
	&& {\Csf_{j}} &&&&&&& {\Csf_{j}}
	\arrow[""{name=0, anchor=center, inner sep=0}, "{F_{jl}}"{description}, from=4-3, to=2-3]
	\arrow[""{name=1, anchor=center, inner sep=0}, "{F_{il}}", from=3-1, to=2-3]
	\arrow["{F_{kl}}"', from=3-5, to=2-3]
	\arrow["{F_{ij}}"', from=3-1, to=4-3]
	\arrow[""{name=2, anchor=center, inner sep=0}, "{F_{jk}}"', from=4-3, to=3-5]
	\arrow[""{name=3, anchor=center, inner sep=0}, "{F_{ik}}"{description}, from=3-8, to=3-12]
	\arrow[""{name=4, anchor=center, inner sep=0}, "{F_{il}}", from=3-8, to=2-10]
	\arrow["{F_{kl}}"', from=3-12, to=2-10]
	\arrow["{F_{ij}}"', from=3-8, to=4-10]
	\arrow["{F_{jk}}"', from=4-10, to=3-12]
	\arrow["{\varepsilon_{ijl}}"', shorten <=7pt, shorten >=14pt, Rightarrow, from=1, to=4-3]
	\arrow["{\varepsilon_{jkl}}", shift left=4, shorten <=7pt, shorten >=7pt, Rightarrow, from=0, to=2]
	\arrow["{\varepsilon_{ikl}}", shorten <=7pt, shorten >=10pt, Rightarrow, from=4, to=3]
	\arrow["{\varepsilon_{ijk}}", shorten <=5pt, shorten >=3pt, Rightarrow, from=3, to=4-10]
\end{tikzcd}$$
that $\varepsilon_{ijl}\circ\varepsilon_{jkl}=\varepsilon_{ijk}\circ\varepsilon_{ikl}$. More explicitly, we have natural transformations 
\begin{center}
\begin{tabular}{c c}
  $\varepsilon_{ijl}:F_{il}\to F_{jl}\circ F_{ij}$ & $\varepsilon_{ikl}:F_{il}\to F_{kl}\circ F_{ik}$ \\
  $\varepsilon_{jkl}:F_{jl}\to F_{kl}\circ F_{jk}$ & $\varepsilon_{ijk}:F_{ik}\to F_{jk}\circ F_{ij}$
\end{tabular}
\end{center}
where the ``cocycle condition'' asserts that the the composition of functors is associative
$$\varepsilon_{jkl}\circ\varepsilon_{ijl}:F_{il}\to(F_{kl}\circ F_{jk})\circ F_{ij}$$
$$\varepsilon_{ijk}\circ\varepsilon_{ikl}:F_{il}\to F_{kl}\circ(F_{jk}\circ F_{ij})$$
by forcing the natural transformations to agree as one would expect. 
\begin{remark}
  $h\Cat_{1}$ is the category whose objects are categories and whose morphisms are functors modulo isomorphic natural transformations. 
\end{remark}
This would imply that in $h\Cat_{1}$ the morphisms are isomorphisms of categories in some appropriate sense. We will characterize isomorphisms of categories using fullness, faithfullness, and essential surjectivity which we now define. 
\begin{definition}[Full Functor]
  A functor $F:\Csf\to\Dsf$ is full if for all $A,B\in\Obj(\Csf)$ the map of sets $\Mor_{\Csf}(A,B)\to\Mor_{\Dsf}(F(A),F(B))$ is surjective.
\end{definition}
\begin{definition}[Faithful Functor]
  A functor $F:\Csf\to\Dsf$ is faithful if for all $A,B\in\Obj(\Csf)$ the map of sets $\Mor_{\Csf}(A,B)\to\Mor_{\Dsf}(F(A),F(B))$ is injective. 
\end{definition}
Naturally, one defines a fully faithful functor as follows. 
\begin{definition}[Fully Faithful Functor]
  A functor $F:\Csf\to\Dsf$ is fully faithful if it is full and faithful, that is for all $A,B\in\Obj(\Csf)$ the map of sets $\Mor_{\Csf}(A,B)\to\Mor_{\Dsf}(F(A),F(B))$ is a bijection. 
\end{definition}
We now define essential surjectivity as follows. 
\begin{definition}[Essentially Surjective Functor]
  A functor $F:\Csf\to\Dsf$ is essentially surjective if for all $B\in\Obj(\Dsf)$ there is an isomorphism $F(X)\to B$ in $\Dsf$. 
\end{definition}
We then define an isomorphism of functors as follows. 
\begin{definition}
  A functor is an isomorphism if it is fully faithful and essentially surjective. 
\end{definition}
We have done a lot of work to think about basic notions such as isomorphisms in new ways. Now let us return to the hunble morphism. Recall that a morphism in a category $\Csf$ is a morphism from the free walking morphism $0\to 1$ to $\Csf$. More explicitly, we have a diagram
$$% https://q.uiver.app/#q=WzAsNCxbMCwwLCJcXE1vcl97XFxDc2Z9KEEsQikiXSxbMCwyLCIqIl0sWzMsMCwiXFxNb3Jfe1xcQ2F0fShbMV0sXFxDc2YpIl0sWzMsMiwiXFxDc2ZcXHRpbWVzXFxDc2YiXSxbMSwzLCIoQSxCKSIsMl0sWzAsMV0sWzIsM10sWzAsMl1d
\begin{tikzcd}
	{\Mor_{\Csf}(A,B)} &&& {\Fun([1],\Csf)} \\
	\\
	{*} &&& \Csf\times\Csf
	\arrow["{(A,B)}"', from=3-1, to=3-4]
	\arrow[from=1-1, to=3-1]
	\arrow[from=1-4, to=3-4]
	\arrow[from=1-1, to=1-4]
\end{tikzcd}$$
taking a morphism $f$ to the tuple of objects in $\Csf$, $(\langle0\rangle^{*}f, \langle1\rangle^{*}f)\in\Obj(\Csf)\times\Obj(\Csf)$. A key notion here is the realization of regular morphisms as functors. This naturally generalizes to the functor category. 
\begin{definition}
  Let $\Csf,\Dsf$ be categories. The functor category $\Fun(\Csf,\Dsf)$ has objects functors $\Csf\to\Dsf$ and morphisms natural transformations between such functors. 
\end{definition}
In the case of simplicial sets, this admits an even nicer description. Let $X,Y$ be simplicial sets. $\Mor_{\SSets}(X,Y)=Y^{X}$ is itself a simplicial set such that maps $Z\to Y^{X}$ can be identified with maps $X\times Z\to Y$. More generally, one can show that for $\Csf,\Dsf$ categories $\Fun(N\Csf, N\Dsf)=N\Dsf^{N\Csf}$ is isomorphic to the nerve of the functor category $N \Fun(\Csf,\Dsf)$. In the case of quasicategories, a theorem of Joyal states that we have the following. 
\begin{theorem}[Joyal]
  If $X$ is a quasicategory and $A$ a simplical set, the functor category $\Fun(A,X)=X^{A}$ is a quasicategory. 
\end{theorem}
\section{Lecture 4 -- 18th September 2023}
We begin with some philosophical ruminations about category theory. Here is a question to stimulate some thought: what would it mean to do math in a quasicategory instead of a category (as we do now)? 
\\\\
Our original plan was to discuss category theory and quasicategory theory in parallel. But to do so will require significant knowledge of category theory in the first place. As with the last class, we devote today to further discussions of notions in ordinary category theory. 
\\\\
Recall our discussion of initial objects (\Cref{def:initial object}) and final objects (\Cref{def:final object}). We denote these $\emptyset$ and $\{*\}$, respectively, since these are the initial and final objects in the category of sets $\Sets$. 
\\\\
Let $I$ be an indexing category, that is a directed graph whose objects are the vertices and morphisms the directed edges. 
\begin{example}
  Both of the following are examples of indexing categories. 
  $$% https://q.uiver.app/#q=WzAsNyxbMCwwLCJcXGJ1bGxldCJdLFsyLDAsIlxcYnVsbGV0Il0sWzAsMSwiXFxidWxsZXQiXSxbMiwxLCJcXGJ1bGxldCJdLFs0LDAsIlxcYnVsbGV0Il0sWzQsMSwiXFxidWxsZXQiXSxbNiwwLCJcXGJ1bGxldCJdLFsxLDNdLFswLDJdLFsyLDNdLFswLDFdLFs0LDZdLFs0LDVdXQ==
  \begin{tikzcd}
    \bullet && \bullet && \bullet && \bullet \\
    \bullet && \bullet && \bullet
    \arrow[from=1-3, to=2-3]
    \arrow[from=1-1, to=2-1]
    \arrow[from=2-1, to=2-3]
    \arrow[from=1-1, to=1-3]
    \arrow[from=1-5, to=1-7]
    \arrow[from=1-5, to=2-5]
  \end{tikzcd}$$
\end{example}
\begin{example}
  Recall the definition of representable spaces of groups \Cref{ex:representable spaces}. Let $BG$ be the representable space of a group $G$. A functor $BG\to\Csf$ is an object of $\Csf$ equipped with a left $G$-action. Indeed, if 
\end{example}
\begin{remark}\label{rmk:top bottom beings}
  What if there were beings on another planet with arms atop and on the bottom of their body. Saying a left action would make absolutely no sense. To be more precise, one ought use the language of contravariant and covariant functors. Suppose we have the following in $BG$
  $$% https://q.uiver.app/#q=WzAsMyxbMCwwLCJcXGJ1bGxldCJdLFsxLDAsIlxcYnVsbGV0Il0sWzIsMCwiXFxidWxsZXQiXSxbMCwxLCJnX3sxfSIsMV0sWzEsMiwiZ197Mn0iLDFdXQ==
  \begin{tikzcd}
    \bullet & \bullet & \bullet
    \arrow["{g_{1}}"{description}, from=1-1, to=1-2]
    \arrow["{g_{2}}"{description}, from=1-2, to=1-3]
  \end{tikzcd}$$
  The image of $BG$ under the functor is an object $g_{2}(g_{1}\cdot x)=(g_{1}g_{2})\cdot x$. 
\end{remark}
We can now define the cone over an indexing category. 
\begin{definition}[Cones]
  Let $F:I\to\Csf$ be a functor for some indexing category $I$. A cone on $F$ is a object $A\in\Obj(\Csf)$ along with maps $g_{i}:F(i)\to A$ for all $i\in\Obj(I)$ such that for all $i\to j\in\Mor_{I}$ the diagram 
  $$% https://q.uiver.app/#q=WzAsMyxbMCwwLCJGKGkpIl0sWzIsMCwiRihqKSJdLFsyLDEsIkEiXSxbMCwxLCJGKGlcXHRvIGopIl0sWzEsMiwiZ197an0iXSxbMCwyLCJnX3tpfSIsMl1d
  \begin{tikzcd}
    {F(i)} && {F(j)} \\
    && A
    \arrow["{F(i\to j)}", from=1-1, to=1-3]
    \arrow["{g_{j}}", from=1-3, to=2-3]
    \arrow["{g_{i}}"', from=1-1, to=2-3]
  \end{tikzcd}$$
  commutes. 
\end{definition}
Let us now look at some examples. 
\begin{example}\label{ex:prepushout}
  Let $I$ be the indexing category/diagram
  $$% https://q.uiver.app/#q=WzAsMyxbMCwwLCJcXGJ1bGxldCJdLFsyLDAsIlxcYnVsbGV0Il0sWzAsMSwiXFxidWxsZXQiXSxbMCwxXSxbMCwyXV0=
  \begin{tikzcd}
    \bullet && \bullet \\
    \bullet
    \arrow[from=1-1, to=1-3]
    \arrow[from=1-1, to=2-1]
  \end{tikzcd}$$
  the cone over the diagram is an object $*\in\Obj(I)$ such that the diagram 
  $$% https://q.uiver.app/#q=WzAsNCxbMCwwLCJcXGJ1bGxldCJdLFsyLDAsIlxcYnVsbGV0Il0sWzAsMSwiXFxidWxsZXQiXSxbMiwxLCIqIl0sWzAsMV0sWzAsMl0sWzEsMywiIiwwLHsic3R5bGUiOnsiYm9keSI6eyJuYW1lIjoiZGFzaGVkIn19fV0sWzIsMywiIiwyLHsic3R5bGUiOnsiYm9keSI6eyJuYW1lIjoiZGFzaGVkIn19fV1d
  \begin{tikzcd}
    \bullet && \bullet \\
    \bullet && {*}
    \arrow[from=1-1, to=1-3]
    \arrow[from=1-1, to=2-1]
    \arrow[dashed, from=1-3, to=2-3]
    \arrow[dashed, from=2-1, to=2-3]
  \end{tikzcd}$$
  commutes. 
\end{example}
\begin{remark}
  In such a situation, we write $F\to*$ for the cone on $F$. 
\end{remark}
\begin{definition}[Colimit Cone]\label{def: colimit cone}
  Let $\Csf$ be a category and $F\to A$ with $A\in\Obj(\Csf)$ for some indexing category $I$. The cone $F\to A$ is a colimit cone if for all other cones $F\to B$ with $B\in\Obj(\Csf)$ the diagram
  $$% https://q.uiver.app/#q=WzAsMyxbMSwwLCJGIl0sWzAsMSwiQSJdLFsyLDEsIkIiXSxbMSwyLCJcXGV4aXN0cyIsMix7InN0eWxlIjp7ImJvZHkiOnsibmFtZSI6ImRhc2hlZCJ9fX1dLFswLDFdLFswLDJdXQ==
  \begin{tikzcd}
    & F \\
    A && B
    \arrow["\exists"', dashed, from=2-1, to=2-3]
    \arrow[from=1-2, to=2-1]
    \arrow[from=1-2, to=2-3]
  \end{tikzcd}$$
  commutes. 
\end{definition}
If $F\to A$ is a colimit cone, we say $A$ is the colimit of $F$ where one writes $$\varprojlim F=\colim_{I}F=A.$$
\begin{example}[Pushout]
  Let $I$ be the indexing category as in \Cref{ex:prepushout}. Let $F:I\to\Csf$ be a functor. The colimit of $F$ is the pushout of the diagram. 
\end{example}
We can actually define colimits differently using universal properties. 
\begin{definition}[Colimit Cone]
  Let $\Csf$ be a category and $F:I\to\Csf$ a functor from some indexing category $I$. Let $F/\Csf$ be the category whose objects are cones on $F$ and whose morphisms are maps $(F\to A)\to(F\to B)$ are morphisms $f\in\Mor_{\Csf}(A,B)$ making the diagram 
  $$% https://q.uiver.app/#q=WzAsMyxbMSwwLCJGIl0sWzAsMSwiQSJdLFsyLDEsIkIiXSxbMSwyLCJmIiwyXSxbMCwxXSxbMCwyXV0=
  \begin{tikzcd}
    & F \\
    A && B
    \arrow["f"', from=2-1, to=2-3]
    \arrow[from=1-2, to=2-1]
    \arrow[from=1-2, to=2-3]
  \end{tikzcd}$$
  commutes. The colimit cone of $F$ is the initial object of the category $F/\Csf$. 
\end{definition}
One can check that the above definitions of the colimit agree, and that the colimit can be described by a universal property, that is, it is unique up to unique isomorphism. \\\\
\emph{Universal properties allow you to dream about an object. -- Emily Riehl}
\begin{definition}[Coequalizer]\label{def:coequalizer}
  Let $I$ be the indexing category 
  $$% https://q.uiver.app/#q=WzAsMixbMCwwLCJcXGJ1bGxldCJdLFsyLDAsIlxcYnVsbGV0Il0sWzAsMSwiZiIsMCx7Im9mZnNldCI6LTF9XSxbMCwxLCJnIiwyLHsib2Zmc2V0IjoxfV1d
  \begin{tikzcd}
    \bullet && \bullet
    \arrow["f", shift left, from=1-1, to=1-3]
    \arrow["g"', shift right, from=1-1, to=1-3]
  \end{tikzcd}$$
  and if $F:I\to\Csf$ is a functor for any category $\Csf$, we define the colimit of $I$ to be the coequalizer of $f$ and $g$. 
\end{definition}
\begin{example}
  If $\Csf=\Sets$ then the coequalizer of the diagram 
  $$% https://q.uiver.app/#q=WzAsMixbMCwwLCJcXGJ1bGxldCJdLFsyLDAsIlxcYnVsbGV0Il0sWzAsMSwiZiIsMCx7Im9mZnNldCI6LTF9XSxbMCwxLCJnIiwyLHsib2Zmc2V0IjoxfV1d
  \begin{tikzcd}
    \bullet && \bullet
    \arrow["f", shift left, from=1-1, to=1-3]
    \arrow["g"', shift right, from=1-1, to=1-3]
  \end{tikzcd}$$
  is $S_{2}/\sim$ where $\sim$ is the equivalence relation generated by $f(s)\sim g(s)$ for all $s\in S_{1}$. 
\end{example}
\begin{example}
  If $\Csf=\AbGrp$ the coequalizer of the diagram 
  $$% https://q.uiver.app/#q=WzAsMixbMCwwLCJcXGJ1bGxldCJdLFsyLDAsIlxcYnVsbGV0Il0sWzAsMSwiZiIsMCx7Im9mZnNldCI6LTF9XSxbMCwxLCJnIiwyLHsib2Zmc2V0IjoxfV1d
  \begin{tikzcd}
    \bullet && \bullet
    \arrow["f", shift left, from=1-1, to=1-3]
    \arrow["g"', shift right, from=1-1, to=1-3]
  \end{tikzcd}$$
  is the cokernel of $A\to B$, that is $B/\mathrm{im}(A)$. 
\end{example}
\begin{example}
  Let $I$ be the indexing category with no non-identity morphisms. THe colimit over $I$ is the coproduct. In $\Sets$ the coproduct is the disjoint union, in $\Grp$ the coproduct is the free product, and in $\AbGrp$ the coproduct is the direct sum. 
\end{example}
One can in fact show the following theorem. 
\begin{theorem}
  Let $\Csf$ be a category. If $\Csf$ has all coproducts and coequalizers, then $\Csf$ has all colimits. 
\end{theorem}
One can consider dual objects by taking the opposite category to define co-cones, limits, etc. \\\\
The guiding philosophy here is that categories and categorical structures can be built out of functors from indexing categories/diagrams. 
\begin{definition}[Constant Functor]
  Let $I$ be an indexing category, $\Csf$ a category, and $F:I\to\Csf$ a functor. For $A\in\Obj(\Csf)$, define the constant functor at $A$ $\delta_{A}:I\to\Csf$ the functor that maps each object of $I$ to $A\in\Obj(\Csf)$ and each morphism to $\id_{A}$. 
\end{definition} 
We can think of a cone on $F$ as a natural transformation $F\to\delta_{A}$. Consider $\Csf^{I}=\Fun(I,\Csf)$ the category of functors from $I$ to $\Csf$. There is a functor $\colim_{I}:\Csf^{I}\to\Csf$ taking a diagram to its colimit. Dually, there is a functor $\delta_{\Obj(\Csf)}:\Csf\to\Csf^{I}$ by $A\mapsto\delta_{A}$. There is a natural transformation $\id_{\Csf^{I}}\to\delta_{\colim_{I}}$. This is the universal property of colimits, that there is an isomorphism in $\Sets$ between $\Mor_{\Csf^{I}}(F,\delta_{A})$ and $\Mor_{\Csf}(\colim_{I}F, A)$. This was our first example of adjoint functors, first defined by Dan Kan. 
\begin{definition}[Adjunction]
  Let $\Csf,\Dsf$ be categories and $F:\Csf\to\Dsf,G:\Dsf\to\Csf$ be functors. An adjunction between $F$ and $G$ is a natural isomorphism of functors $\Csf^{\Opp}\times\Dsf\to\Sets$ by $\Mor_{\Dsf}(F(X),Y)\to\Mor_{\Csf}(X,F(Y))$ for all $X\in\Obj(\Csf), Y\in\Obj(\Dsf)$. 
\end{definition}
This leads to several closely related notions. 
\begin{definition}[Adjoint Pair]
  Let $\Csf,\Dsf$ be categories and $F:\Csf\to\Dsf,G:\Dsf\to\Csf$ be functors. If there is an adjunction between $F$ and $G$ then we say $F$ and $G$ form an adjoint pair. 
\end{definition}
\begin{definition}[Left and Right Adjoints]
  Let $\Csf,\Dsf$ be categories and $F:\Csf\to\Dsf,G:\Dsf\to\Csf$ be an adjoint pair. We say $F$ is the left adjoint and $G$ is the right adjoint. 
\end{definition}
\begin{remark}
  The language of left and right adjoints can be confusing. See, for example, the remark above. The directionality could be deduced from the following diagram, which is often used in the literature. 
  $$% https://q.uiver.app/#q=WzAsMixbMCwwLCJGOlxcQ3NmIl0sWzIsMCwiXFxEc2Y6RyJdLFsxLDAsIiIsMCx7Im9mZnNldCI6LTF9XSxbMCwxLCIiLDAseyJvZmZzZXQiOi0xfV1d
  \begin{tikzcd}
    {F:\Csf} && {\Dsf:G}
    \arrow[shift left, from=1-3, to=1-1]
    \arrow[shift left, from=1-1, to=1-3]
  \end{tikzcd}$$
\end{remark}
We revisit colimits in a new language. 
\begin{example}
  Let $\Csf$ be a category, $I$ an indexing category, and $F:I\to\Csf$ a functor. The colimit functor $\colim_{I}:\Csf^{I}\to\Csf$ and $\delta_{\Obj(\Csf)}:\Csf\to\Csf^{I}$ form an adjoint pair. The functor $\colim_{I}:\Csf^{I}\to\Csf$ is the left adjoint to $\delta_{\Obj(\Csf)}:\Csf\to\Csf^{I}$. Analogously, $\delta_{\Obj(\Csf)}:\Csf\to\Csf^{I}$ is the right adjoint to $\colim_{I}:\Csf^{I}\to\Csf$. 
\end{example}
\begin{remark}
  Dan Kan began writing the paper on adjoint functors without full knowledge of how it would develop. The lesson here is to start writing early, even when you don't think something will be significant. 
\end{remark}
One can easily deduce the following about adjunctions. 
\begin{proposition}
  Suppose we have categories $\Csf,\Dsf,\Esf$ and functors $F_{1},F_{2},G_{1},G_{2}$ as below where $F_{1},G_{1}$ and $F_{2},G_{2}$ are adjoint pairs. The pair $F_{2}\circ F_{1}, G_{1}\circ G_{2}$ is an adjoint pair as well. 
  $$% https://q.uiver.app/#q=WzAsMyxbMCwwLCJcXENzZiJdLFsyLDAsIlxcRHNmIl0sWzQsMCwiXFxFc2YiXSxbMCwxLCJGX3sxfSIsMCx7Im9mZnNldCI6LTF9XSxbMSwyLCJGX3syfSIsMCx7Im9mZnNldCI6LTF9XSxbMiwxLCJHX3syfSIsMCx7Im9mZnNldCI6LTF9XSxbMSwwLCJHX3sxfSIsMCx7Im9mZnNldCI6LTF9XV0=
  \begin{tikzcd}
    \Csf && \Dsf && \Esf
    \arrow["{F_{1}}", shift left, from=1-1, to=1-3]
    \arrow["{F_{2}}", shift left, from=1-3, to=1-5]
    \arrow["{G_{2}}", shift left, from=1-5, to=1-3]
    \arrow["{G_{1}}", shift left, from=1-3, to=1-1]
  \end{tikzcd}$$
\end{proposition}
\begin{proof}
  We have natural isomorphisms 
  $$\Mor_{\Esf}(F_{2}(F_{1}(X)),Y)\to\Mor_{\Dsf}(F_{1}(X),G_{2}(Y))\to\Mor_{\Csf}(X,G_{2}(G_{1}(Y)))$$
  for all $X\in\Obj(\Csf),Y\in\Obj(\Esf)$ proving the claim. 
\end{proof}
We now introduce sheaves. One likely encounters this in an algebraic geometry course. 
\begin{example}[Sheaves on a Space]
  Let $X$ be a topological space and $X^{\Opens}$ be the category of open sets of $X$ whose objects are open sets and whose morphisms are inclusion maps. A $\Csf$-valued sheaf on $X$ is a contravariant functor $\Fun((X^{\Opens})^{\Opp},\Csf)$ from the category of open sets on $X$ to $\Csf$. 
\end{example}
We now define sheaves on categories, a generalization of the abovementioned concept. 
\begin{definition}[Sheaves on Categories]
  Let $\Csf$ be a category. The category of $\Dsf$-valued presheaves on $\Csf$ is the functor category $\Fun(\Csf^{\Opp},\Dsf)=\PSh(\Csf)$. 
\end{definition}
\begin{remark}
  The notation $\PSh(\Csf)$ does not indicate the category in which the presheaf is valued. This is often obvious from context. If this is not indicated, we will take $\Dsf=\Sets$. 
\end{remark}
Naturally, there is a map from the (higher) category of categories $\Cat$ taking a category $\Csf$ to the category of presheaves on it $\PSh(\Csf)$. 
\begin{definition}[Representable Presheaves]
  A presheaf on $\Csf$ is representable if it is naturally isomorphic to $\Fun(-,\Csf)$.
\end{definition}
One can then show that this is a category that fulfils several nice properties. 
\begin{theorem}
  Let $\Csf$ be a category. The category $\PSh(\Csf)$ admits all limits and colimits, and every $F\in\Obj(\PSh(\Csf))$ is the colimit of representable functors. 
\end{theorem}
\section{Lecture 5 -- 20th September 2023}
We want to define adjoints, initial and final objects, and analogous constructions from ordinary category theory in the setting of quasicategories. Unfortunately, technical stuff gets in the way. Let's try and wade through this today. 
\\\\
Let $I$ be a directed graph and let $\Csf^{I}=\Fun(I,\Csf)$ be the category whose objects are functors $I\to\Csf$ and whose morphisms are natural transformations between such functors. We will now see how $\Fun(I,\Csf)$ is naturally isomorphic to $\Mor_{\SSets}(NI,N\Csf)$. 
\begin{proposition}
  There exists a natural isomorphism $\Fun(I,\Csf)\to\Mor_{\SSets}(NI,N\Csf)$. 
\end{proposition}
% Proof
\newpage
\printbibliography
\end{document}